\documentclass[a4paper,14pt]{article} %размер бумаги устанавливаем А4, шрифт 12пунктов
\usepackage[T2A]{fontenc}
\usepackage[utf8]{inputenc}
\usepackage[english,russian]{babel} %используем русский и английский языки с переносами
\usepackage{amssymb,amsfonts,amsmath,mathtext,enumerate,float,amsthm} %подключаем нужные пакеты расширений
\usepackage[pdftex,unicode,colorlinks=true,citecolor=black,linkcolor=black]{hyperref}
%\usepackage[pdftex,unicode,colorlinks=true,linkcolor=blue]{hyperref}
\usepackage{indentfirst} % включить отступ у первого абзаца
\usepackage[dvips]{graphicx} %хотим вставлять рисунки?
\graphicspath{{illustr/}}%путь к рисункам

\makeatletter
\renewcommand{\@biblabel}[1]{#1.} % Заменяем библиографию с квадратных скобок на точку:
\makeatother %Смысл этих трёх строчек мне непонятен, но поверим "Запискам дебианщика"

\usepackage{geometry} % Меняем поля страницы.
\geometry{left=4cm}% левое поле
\geometry{right=1cm}% правое поле
\geometry{top=2cm}% верхнее поле
\geometry{bottom=2cm}% нижнее поле

\renewcommand{\theenumi}{\arabic{enumi}}% Меняем везде перечисления на цифра.цифра
\renewcommand{\labelenumi}{\arabic{enumi}}% Меняем везде перечисления на цифра.цифра
\renewcommand{\theenumii}{.\arabic{enumii}}% Меняем везде перечисления на цифра.цифра
\renewcommand{\labelenumii}{\arabic{enumi}.\arabic{enumii}.}% Меняем везде перечисления на цифра.цифра
\renewcommand{\theenumiii}{.\arabic{enumiii}}% Меняем везде перечисления на цифра.цифра
\renewcommand{\labelenumiii}{\arabic{enumi}.\arabic{enumii}.\arabic{enumiii}.}% Меняем везде перечисления на цифра.цифра

\sloppy


\renewcommand\normalsize{\fontsize{14}{25.2pt}\selectfont}

\usepackage[backend=biber,style=gost-numeric,sorting=none]{biblatex}
\addbibresource{../common/notmy.bib}


\begin{document}
\renewcommand{\bibname}{Список цитированной литературы}
\renewcommand\refname{\bibname}
% !!!
% Здесь начинается реальный ТеХ-код
% Всё, что выше - беллетристика

\paragraph{Определение.}
Системой Эрдёша называется множество точек $\{M_1, M_2, ..., M_n\}$ на плоскости, не содержащееся ни в какой прямой,
такое, что для любых $i\neq j$ расстояние $|M_i M_j| \in \mathbb{N}$,
т.е. является натуральным числом.

\paragraph{Лемма 1.}

Для любой системы Эрдёша $S=\{M_1, M_2, ..., M_n\}$ существует такое целое число $m$, не превосходящее её диаметра $diam(S)$,
что можно выбрать систему координат способом, при котором координаты каждой точки системы будут иметь вид
\begin{equation}
	M_i = \left(
		\frac{p_i}{2m}
		;
		\frac{\pm\sqrt{q_i}}{2m}
	\right),
\end{equation}
где $p_i \in \mathbb{Z}, q_i \in \mathbb{N}$.


\paragraph{Доказательство.}
Пусть $|M_1 M_2| = m$, тогда, очевидно, $m \leq diam(S)$.
Введём систему координат так, что $M_1=(-m/2, 0)$, $M_2=(m/2, 0)$
Рассмотрим некоторую другую точку $M_i=(x, y)\in S$.
Обозначим $M_i M_1 = a$, $M_i M_2 = b$.
Тогда $M_i$ лежит на пересечении двух окружностей,
задаваемых уравнениями
\begin{gather}
	\left(-\frac{m}{2} - x\right)^2 + y ^2 = a^2,
\\
	\left( \frac{m}{2} - x\right)^2 + y ^2 = b^2.
\end{gather}

Решив её, имеем
\begin{gather}
	x = \frac{a^2 - b^2}{2 m} = \frac{p_i}{2m},
\\
%	y = \frac{\pm \sqrt{-(a^2 - b^2)^2 + m^2 (2 a^2 + 2 b^2  - m^2)} }{2m}.
	y = \pm\sqrt{a^2 - \left(\frac{m}{2}+x\right)^2} =
	\pm\sqrt{a^2 - \left(\frac{m}{2}+\frac{p_i}{2m}\right)^2} =
	\frac{\pm\sqrt{q_i}}{2m},
\end{gather}
что и требовалось доказать.

\paragraph{Лемма 2.}

Для любой системы Эрдёша $S=\{M_1, M_2, ..., M_n\}$ существует такое целое число $m$, не превосходящее её диаметра $diam(S)$,
и такое натуральное число $q$, свободное от квадратов (ссылка!!),
что можно выбрать систему координат способом, при котором координаты каждой точки системы будут иметь вид
\begin{equation}
	M_i = \left(
		\frac{p_i}{2m}
		;
		\frac{s_i\sqrt{q}}{2m}
	\right),
\end{equation}
где $p_i, s_i \in \mathbb{Z}$.

\paragraph{Доказательство.}
Рассмотрим две точки $M_1, M_2 \in S$.
По лемме 1 их координаты (при введении соответствующей системы координат) выражаются в виде
\begin{equation}
	M_1 = \left(
		\frac{p_1}{2m}
		;
		\frac{\pm\sqrt{q_1}}{2m}
	\right),
	M_2 = \left(
		\frac{p_2}{2m}
		;
		\frac{\pm\sqrt{q_2}}{2m}
	\right),
\end{equation}
где $p_i \in \mathbb{Z}, q_i \in \mathbb{N}$, $i=1,2$.

Тогда
\begin{equation}
	|M_1 M_2| = \frac{1}{2m} \sqrt{p_1^2 + p_2^2 - 2p_1p_2 + q_1 + q_2 - 2\sqrt{q_1 q_2}}
\end{equation}
Для того, чтобы расстояние $|M_1 M_2| \in \mathbb{N}$, необходимо,
чтобы $|M_1 M_2|^2 \in \mathbb{N}$, для чего, в свою очередь, необходимо, чтобы
$p_1^2 + p_2^2 - 2p_1p_2 + q_1 + q_2 - 2\sqrt{q_1 q_2} \in \mathbb{N}$,
что влечёт $\sqrt{q_1 q_2} \in \mathbb{N}$,
что, в свою очередь, требует соотношения
$q_1 = s_1^2 \sqrt{q}$, $q_2 = s_2^2 \sqrt{q}$.
Лемма доказана.

\paragraph{Лемма 3.}
Число $q$ в лемме 2 не зависит от выбора <<опорных>> точек для системы координат в лемме 1.

\paragraph{Доказательство.}
Покажем для простоты, что при замене <<опорной>> точки $M_2$ на $M_3$ число $q$ сохраняется.
Пусть в системе координат с <<опорными>> точками $M_1$ и $M_2$ имеем
$M_1=(x_1, 0)$, $M_2=(x_2, 0)$, $M_3=(x_3, y_3)$,
где $x_1, x_2, x_3$~--- рациональные, $y_3 = \frac{s_3\sqrt{q}}{2m}$ по лемме 2;
и пусть в системе координат с <<опорными>> точками $M_1$ и $M_3$ имеем
$M_1=(\tilde{x}_1, 0)$, $M_3=(\tilde{x}_3, 0)$, $M_2=(\tilde{x}_2, \tilde{y}_2)$,
где $\tilde{x}_1, \tilde{x}_2, \tilde{x}_3$~--- рациональные, $\tilde{y}_2 = \frac{s_2\sqrt{\tilde{q}}}{2\tilde{m}}$ по лемме 2.
Тогда площадь треугольника $M_1 M_2 M_3$ двумя способами выражается как полупроизведение основания на высоту:
\begin{equation}
	\frac{1}{2} |y_3| \cdot |x_1 - x_2|
	=
	\frac{1}{2} |\tilde{y}_2| \cdot |\tilde{x}_1 - \tilde{x}_3|,
\end{equation}
откуда имеем
\begin{equation}
	|y_3|
	=
	|\tilde{y}_2| \frac{ |\tilde{x}_1 - \tilde{x}_3|}{|x_1 - x_2|},
\end{equation}
или, подставив выражения для $\tilde{y}_2$ и $y_3$,
\begin{equation}
	\left|\frac{s_3\sqrt{q}}{2m}\right|
	=
	\left|\frac{s_2\sqrt{\tilde{q}}}{2\tilde{m}}\right| \frac{ |\tilde{x}_1 - \tilde{x}_3|}{|x_1 - x_2|},
\end{equation}

Сгруппировав иррациональные множители в левой части, а рациональные~--- в правой, получим
\begin{equation}
	\left|\frac{\sqrt{q}}{\sqrt{\tilde{q}}}\right|
	=
	\left|\frac{s_2}{2\tilde{m}}\right| \frac{ |\tilde{x}_1 - \tilde{x}_3|\cdot|2m|}{|s_3|\cdot|x_1 - x_2|},
\end{equation}
или, что то же самое,
\begin{equation}
	\sqrt{\frac{q}{\tilde{q}}}
	=
	\left|\frac{s_2}{2\tilde{m}}\right| \frac{ |\tilde{x}_1 - \tilde{x}_3|\cdot|2m|}{|s_3|\cdot|x_1 - x_2|},
\end{equation}

Так как $q$ b $\tilde{q}$ свободны от квадратов, то $\frac{q}{\tilde{q}} = 1$, т.е. $q = \tilde{q}$,
что и требовалось доказать.


\printbibliography

\end{document}
