\documentclass[a4paper,openbib]{article}
\pagestyle{empty} % нумерация выкл.
\usepackage{amsmath}
\usepackage[utf8]{inputenc}
\usepackage[english,russian]{babel}
\usepackage{amsfonts,amssymb}
\usepackage{latexsym}
\usepackage{euscript}
\usepackage{enumerate}
\usepackage{graphics}
\usepackage[dvips]{graphicx}
\usepackage{geometry}
\graphicspath{{examples/}}
\usepackage{wrapfig}
\usepackage[colorlinks=true,urlcolor=blue]{hyperref}
\usepackage{indentfirst}

\righthyphenmin=2

\usepackage[14pt]{extsizes}

\geometry{left=2.4cm}% левое поле
\geometry{right=2.4cm}% правое поле
\geometry{top=2.2cm}% верхнее поле
\geometry{bottom=3.2cm}% нижнее поле

\renewcommand{\baselinestretch}{1.3}

\renewcommand{\leq}{\leqslant}
\renewcommand{\geq}{\geqslant}

\newcommand{\longcomment}[1]{}

\begin{document}
\noindent УДК 514.112.3
%\clubpenalty=10000
%\widowpenalty=10000


\begin{center}
	\bfО ХАРАКТЕРИСТИКЕ ПЛОСКИХ МНОЖЕСТВ С ЦЕЛОЧИСЛЕННЫМИ РАССТОЯНИЯМИ\\С РЁБРАМИ 1 И 2
	\footnote{
		Работа выполнена в Воронежском университете при поддержке РНФ, грант 19-11-00197.
	}

	\bfЛушина Е. А., Авдеев Н. Н.

	\itВоронежский Государственный Университет
\end{center}

Аннотация:

Ключевые слова:
	плоские множества с целочисленными расстояниями,
	характеристика треугольника,
	уравнение Пелля

Abstract:

Keywords:
	planar integral point sets,
	characteristic of a triangle,
	Pell equation




Обозначим через $\mathfrak{M}$ множество таких подмножеств $M\left\{M_{1}, M_{2}, \ldots\right\} \subset$ $\mathbb{R}^{2}$, что $\left|M_{i}-M_{j}\right| \in \mathbb{N}$ для всех $\mathrm{i}, \mathrm{j}$, где $\mathbb{N}-$ множество натуральных чисел и $\left|M_{i}-M_{j}\right|-$ расстояние между точками $M_{i}$ и $M_{j} .$




Доказано [1,2], что всякое бесконечное множество $M \in \mathfrak{M}$ содержится в некоторой прямой $L \subset \mathbb{R}^{2}$, и поэтому оно является предметом изучения не столько комбинаторной геометрии, сколько теории чисел, что выходит за рамки рассматриваемой проблемы.

Определение. Мощностью множества М, состоящего из конечного
числа элементов, называется количество элементов этого множества. Обозначается |M|.

Для заданного $n \in \mathbb{N}$ обозначим через $\mathfrak{M}_{n}$ множество таких $M \in \mathfrak{M}$, что |M|=n и $M \not \subset L$ для любой прямой $L \subset \mathbb{R}^{2}$ Множество $M \in \mathfrak{M}$ характеризуется такими параметрами, как
1. Мощность
2. Диаметр
3. Характеристика
4. Минимальное ребро Определение. Диаметром множества М будем называть наибольшее расстояние между любыми двумя точками множества М.
$\operatorname{diam} M=\max |A-B|$, где $A, B \in M$
Определение. Характеристикой множества $M \in \mathfrak{M}_{n} \quad$ называется свободное от квадратов число р такое, что плошадь любого треугольника $\mathrm{ABC}$, где $A, B, C \in M$ соизмерима $\mathrm{c} \sqrt{p} .$ Обозначается char $\mathrm{M}=\mathrm{p} .$

Определение. Ребром множества М будем называть любой отрезок $M_{1} M_{2}$, где $M_{1}, M_{2} \in M$
Множество М характеристики р может быть размещено на решетке
$$
\left\{\left(\frac{a_{i}}{2 m} ; \frac{b_{i} \sqrt{p}}{2 m}\right)\right\}
$$
где $a_{i}, b_{i} \in \mathbb{Z}(\mathbb{Z}-$ множество целых чисел), в качестве $\mathrm{m}$ можно взять длину любого ребра множества $\mathrm{M}$ [3, Теорема 4].
Приведем несколько известных фактов о множествах $M \in \mathfrak{M}_{n}:$
\begin{itemize}
- Для любого $n \in \mathbb{N}, n \geq 3$ выполнено $\mathfrak{M}_{n} \neq \emptyset$ [3, Теорема 2].
$>\quad \mathrm{C}$ возрастанием мощности диаметр возрастает не менее чем линейно [4,6].
- Для любой мощности $n \in \mathbb{N}, n \geq 3$ и любого свободного от квадратов числа р существует множество $M \in \mathfrak{M}_{n}$ такое, что char $\mathrm{M}=\mathrm{p}[3,$, Teopema 5 ].

\item Для любой мощности $n \in \mathbb{N}, n \geq 3$ и любого натурального числа $\mathrm{k}$ существует множество $M \in \mathfrak{M}_{n}$, содержащее ребро $\mathrm{k}[5] .$
$>\quad$ Не существует множества $M \in \mathfrak{M}_{n}$ характеристики 1 с ребром 1
$[3] .$
\end{itemize}
В статье «О существовании специальных множеств с целочисленными расстояниями с ребром 1» получены следующие результаты исследования множества с характеристикой 2,3 и $5:$
1. $\quad$ Не существует множества $M \in \mathfrak{M}_{3}$ характеристики 2 или $5 \mathrm{c}$ ребром $1 .$
2. $\quad$ Существует бесконечное семейство множеств $M_{i} \in \mathfrak{M}_{3}$, char M=3
с ребром 1 и char $\mathrm{M}=5$ с ребром $2 .$ В данной работе рассматривается множество $M \in \mathfrak{M}_{3}$, char M=6 и char

М=7 с ребром 1. Доказывается, что не существует множества $M \in \mathfrak{M}_{3} \mathrm{c}$ характеристикой вида $4 k+1,4 k+2$, при $k=0,1,2, \ldots$ Вместе с тем, мы строим бесконечное семейство множеств $M_{i} \in \mathfrak{M}_{3}$, char $\mathrm{M}=4 k+3$, при $k=$ $0,1,2, \ldots$ с ребром $1 .$
Утверждение 1. Не существует множества $M \in \mathfrak{M}_{3}$ характеристики $6 \mathrm{c}$ ребром $1 .$ Доказательство. По теореме о сетке введем систему координат так, что $\mathrm{O}(0 ; 0), A\left(-\frac{1}{2} ; 0\right), B\left(\frac{1}{2} ; 0\right), C\left(0 ; \frac{b \sqrt{6}}{2}\right)$ и $A, B, \mathrm{C} \in M .$ По теореме
Пифагора:
$$
\begin{array}{c}
\left(\frac{1}{2}\right)^{2}+\left(\frac{b \sqrt{6}}{2}\right)^{2}=r^{2} \\
1+6 b^{2}=4 r^{2}
\end{array}
$$
Очевидно, что в уравнении (1) левая часть - нечетное число при любом $b \in \mathbb{Z}$, а правая $-$ четное число при любом $r \in \mathbb{Z} .$ Следовательно, уравнение
(1) неразрешимо в целых числах. $\mathbf{y}_{\text {тверждение }}$ 2. Существует бесконечное семейство множеств $M_{i} \in \mathfrak{M}_{n}$ характеристики 7 с ребром $1 .$

Доказательство. По теореме о сетке введем систему координат так, что $\mathrm{O}(0 ; 0), A\left(-\frac{1}{2} ; 0\right), B\left(\frac{1}{2} ; 0\right), C\left(0 ; \frac{\mathrm{b} \sqrt{7}}{2}\right)$ и $A, B, \mathrm{C} \in M .$ По теореме
Пифагора:
$$
\begin{array}{c}
\left(\frac{1}{2}\right)^{2}+\left(\frac{b \sqrt{7}}{2}\right)^{2}=r^{2} \\
1+7 b^{2}=4 r^{2}
\end{array}
$$
Уравнение (2) имеет решение в целых числах, если $\left(\frac{1+7 b^{2}}{4}\right)-$ полный квадрат. Например, $b^{2}=9, \mathrm{r}^{2}=16 .$ Таким образом, $(4 ; 3)-$ решение уравнения (2).
Уравнение (2) равносильно следующему уравнению:
$4 r^{2}-7 b^{2}=1$
Введем замену: $n=2 r .$ Получим
$$
n^{2}-7 b^{2}=1
$$
Уравнение (4) называется уравнением Пелля. Определение. Уравнение Пелля - диофантово уравнение вида
$$
x^{2}-m y^{2}=1
$$
где $\mathrm{m}-$ натуральное число, не являющееся квадратом. Ясно, что уравнение Пелля имеет тривиальное решение $\mathrm{x}=\pm 1, \mathrm{y}=0 .$ Теорема 1. Любое уравнение Пелля имеет нетривиальное решение [7]. Теорема 2. Все нетривиальные положительные решения получаются многократным умножением основного решения на себя [7].

Применим формулу разности квадратов к уравнению (4) и подставим вместо неизвестных найденное решение:
$$
\begin{array}{l}
(n-b \sqrt{7})(n+b \sqrt{7})=1 \\
(8-3 \sqrt{7})(8+3 \sqrt{7})=1
\end{array}
$$
Согласное Теореме 2 :
$$
(8-3 \sqrt{7})^{\mathrm{k}}(8+3 \sqrt{7})^{\mathrm{k}}=1^{\mathrm{k}} \quad \mathrm{k} \in \mathbb{N}
$$
Заметим, что при возведении в четные степени \textrm{K} , ~ п о л у ч а ю т с я ~ н е ч е т н ы е ~

n, а следовательно, $r-$ нецелые числа. Поэтому будем возводить выражение
(5) в степени вида $\mathrm{k}=2 \mathrm{~m}+1, \mathrm{~m}=0,1,2 \ldots$
Приведем несколько решений уравнения (4):
При $\mathrm{k}=3:(2024-765 \sqrt{7})(2024+765 \sqrt{7})=1 \quad$ пара $\quad$ чисел
(2024;765) является решением уравнения (4), следовательно, (1012;765) решение уравнения (3).
$$
\text { При } \mathrm{k}=5:(514088-194307 \sqrt{7})(514088+194307 \sqrt{7})=1
$$
(514088; 194307) - решением уравнения (4), (257044; 194307) - решение уравнения (3).

Продолжая возводить в нечетные степени обе части уравнения, получим бесконечную последовательность решений для исходного уравнения (2):
$$
\left(\frac{x_{n}}{2} ; y_{n}\right), \text { где } x_{n}+y_{n} \sqrt{7}=(8-3 \sqrt{7})^{2 n+1}, n=0,1,2 \ldots
$$
Таким образом, существует бесконечное семейство множеств $M_{i} \in \mathfrak{M}_{n}$ характеристики 7 с ребром $1 .$

Утверждение 3. Всякое множество $M \in \mathfrak{M}_{3}$ с минимальным ребром имеет характеристику вида $\mathrm{p}=4 \mathrm{k}+3, \mathrm{k}=0,1,2 \ldots$
Доказательство. Пусть минимальное ребро множества $\mathfrak{M}_{3}$ равно $1 .$ Воспользуемся теоремой о сетке и введем систему координат так, что О (0;0),
$A\left(-\frac{1}{2} ; 0\right), B\left(\frac{1}{2} ; 0\right), C\left(0 ; \frac{b \sqrt{p}}{2}\right)$ и $A, B, \mathrm{C} \in M .$ По теореме Пифагора:
$$
\begin{array}{c}
\left(\frac{1}{2}\right)^{2}+\left(\frac{\mathrm{b} \sqrt{\mathrm{p}}}{2}\right)^{2}=\mathrm{r}^{2} \\
1+p b^{2}=4 r^{2}
\end{array}
$$
Для того, чтобы $\mathrm{r}^{2}$ было целым числом необходимо, чтобы левая часть уравнения была кратна 4 и переменные $\mathrm{p}, \mathrm{b}-$ нечетные числа. Следовательно, произведение $p b^{2}$ должно удовлетворять следующему условию: $p b^{2}=4 k+$ $+3, k=0,1,2, \ldots$ Тогда $b^{2}=(2 \mathrm{~m}+1)^{2}=4 \mathrm{~m}^{2}+4 \mathrm{~m}+1=4\left(\mathrm{~m}^{2}+\mathrm{m}\right)+1=$
$4 n+1, \mathrm{n}=0,1,2, \ldots$ Так как это равенство выполнено при любом $\mathrm{n}$, оно выполнено и при $\mathrm{n=0.}$ Отсюда, $p=4 k+3, k=0,1,2, \ldots$ $\mathbf{y}_{\text {тверждение } 4 . \text { Не существует множества } M \in \mathfrak{M}_{3} \mathrm{c} \text { характеристикой }}$
$4 k+1, k=0,1,2, \ldots$ и ребром $1 .$
Доказательство. По теореме о сетке введем систему координат так, что $\mathrm{O}(0 ; 0), A\left(-\frac{1}{2} ; 0\right), B\left(\frac{1}{2} ; 0\right), C\left(0 ; \frac{b \sqrt{4 k+1}}{2}\right)$ и $A, B, \mathrm{C} \in M .$ По теореме
Пифагора:
$$
\begin{array}{c}
\left(\frac{1}{2}\right)^{2}+\left(\frac{b \sqrt{4 k+1}}{2}\right)^{2}=r^{2} \\
1+(4 k+1) b^{2}=4 r^{2}
\end{array}
$$
Заметим, что обе части уравнения (5) должны быть одной четности, следовательно, $b=2 \mathrm{~m}+1 .$ Тогда $b^{2}=4 n+1, \mathrm{n}=0,1,2, \ldots$
$$
\begin{aligned}
1+(4 k+1)(4 n+1) &=4 r^{2} \\
& 16 n k+4 k+4 n+2=4 r^{2}
\end{aligned}
$$
Очевидно, что левая часть уравнения (6) ни при каком $n, k \in \mathbb{Z}$ не делится на 4. Таким образом, уравнение (5) неразрешимо в целых числах. $\mathbf{y}_{\mathbf{T в е р} \boldsymbol{ж}}$ ение 4. Не существует множества $M \in \mathfrak{M}_{3} \mathrm{c}$ характеристикой $4 k+2, k=0,1,2, \ldots$ и ребром $1 .$
Доказательство. Воспользуемся способом задания системы координат, указанным в утверждении $3 .$
$$
\begin{array}{c}
\left(\frac{1}{2}\right)^{2}+\left(\frac{b \sqrt{4 k+2}}{2}\right)^{2}=r^{2} \\
1+(4 k+2) b^{2}=4 r^{2}
\end{array}
$$
Заметим, что правая часть уравнения (7) четна при любом $r \in \mathbb{Z}, \mathrm{a}$ левая часть - нечетна при любом $b \in \mathbb{Z} .$ Таким образом, уравнение (7) не имеет решений в целых числах. Утверждение 5. Для всякой характеристики р найдется система с ребром 2
Доказательство. По теореме о сетке введем следующую систему
координат: O $(0 ; 0), A(-1 ; 0), B(1 ; 0), C\left(0 ; \frac{b \sqrt{\mathrm{p}}}{4}\right)$ и $A, B, \mathrm{C} \in M .$ По теореме
Пифагора:
$$
1+\left(\frac{\mathrm{b} \sqrt{\mathrm{p}}}{4}\right)^{2}=\mathrm{r}^{2}
$$
Отсюда,
$$
1+\frac{\mathrm{pb}^{2}}{16}=\mathrm{r}^{2}
$$
Правая часть уравнения (8) является целым числом в том случае, если b=4n, $\mathrm{n}=0,1,2, \ldots$ Тогда
$$
\begin{array}{l}
1+\mathrm{pn}^{2}=r^{2} \\
r^{2}-\mathrm{pn}^{2}=1
\end{array}
$$
Очевидно, равенство (9) является уравнением Пелля. По теореме 1 [7], уравнение (9) имеет нетривиальное решение, при любом свободном оТ квадратов числе $\mathrm{p} .$

Список использованной литературы:
1. Erdös P. Integral distances // Bull. Amer. Math. Soc. --- 1945. --- Vol. 51, N 12 . ---
P. 996.

2. Anning N. H., Erdös P. Integral distances // Bull. Amer. Math. Soc.-1945.Vol. 51, № 8.-P. 598-600.

3. Авдеев Н.Н., Семенов Е.М. Множества точек с целочисленными расстояниями на плоскости и в Евклидовом пространстве.

4. Solymosi J. Note on integral distances //Discrete \& Computational Geometry. --- 2003. --- T. 30. --- №. 2. --- C. 337-342.

5. Зволинский P. E. Facher integral point sets with particular distances of arbitrary cardinality.
В сборнике: Актуальные проблемы прикладной математики, информатики и механики сборник трудов Международной научной конференции. 2021. С. 668-674

6. Avdeev N. N. On existence of integral point sets and their diameter bounds //Australasian Journal of Combinatorics. – 2020. – Т. 77. – С. 100-116.

7. Бугаенко В. О. Уравнение Пелля. Серия: «Библиотека «Математическое просвещение»».М.: МЦНМО, 2001. --- 32с.: ил.


On characteristic of planar integral point sets with edges 1 and 2

{\bf Лушина Екатерина Александровна}, студент математического факультета ФГБОУ ВО «Воронежский государственный университет», г.~Воронеж.

{\bf Авдеев Николай Николаевич}, аспирант кафедры теории функций и геометрии математического факультета ФГБОУ ВО «Воронежский государственный университет», г.~Воронеж.



e-mail: nickkolok@mail.ru

Научный руководитель:
{\bf Семенов Евгений Михайлович},
заведующий кафедрой теории функций и геометрии математического факультета ФГБОУ ВО «Воронежский государственный
университет», г. Воронеж.

\end{document}

