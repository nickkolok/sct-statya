\documentclass[a4paper,openbib]{article}
\pagestyle{empty} % нумерация выкл.
\usepackage{amsmath}
\usepackage[utf8]{inputenc}
\usepackage[english,russian]{babel}
\usepackage{amsfonts,amssymb}
\usepackage{latexsym}
\usepackage{euscript}
\usepackage{enumerate}
\usepackage{graphics}
\usepackage[dvips]{graphicx}
\usepackage{geometry}
\graphicspath{{examples/}}
\usepackage{wrapfig}
\usepackage[colorlinks=true,urlcolor=blue]{hyperref}
\usepackage{indentfirst}

\righthyphenmin=2

\usepackage[14pt]{extsizes}

\geometry{left=2.4cm}% левое поле
\geometry{right=2.4cm}% правое поле
\geometry{top=2.2cm}% верхнее поле
\geometry{bottom=3.2cm}% нижнее поле

\renewcommand{\baselinestretch}{1.3}

\renewcommand{\leq}{\leqslant}
\renewcommand{\geq}{\geqslant}

\newcommand{\longcomment}[1]{}

\begin{document}
\noindent УДК 514.112.3
%\clubpenalty=10000
%\widowpenalty=10000


\begin{center}
	\bfО ХАРАКТЕРИСТИКЕ ПЛОСКИХ МНОЖЕСТВ С ЦЕЛОЧИСЛЕННЫМИ РАССТОЯНИЯМИ\\С РЁБРАМИ 1 И 2
	\footnote{
		Работа выполнена в Воронежском университете при поддержке РНФ, грант 19-11-00197.
	}

	\bfЛушина Е. А., Авдеев Н. Н.

	\itВоронежский Государственный Университет
\end{center}

Аннотация:

Ключевые слова:
	плоские множества с целочисленными расстояниями,
	характеристика треугольника,
	уравнение Пелля

Abstract:

Keywords:
	planar integral point sets,
	characteristic of a triangle,
	Pell equation




Обозначим через $\mathfrak{M}$ множество таких подмножеств $M=\left\{M_{1}, M_{2}, \ldots\right\} \subset$ $\mathbb{R}^{2}$,
что $\left|M_{i}-M_{j}\right| \in \mathbb{N}$ для всех $\mathrm{i}, \mathrm{j}$,
где $\mathbb{N}$ "--- множество натуральных чисел,
$\mathbb{R}^2$ "--- евклидова плоскость
и $\left|M_{i}-M_{j}\right|$ "--- расстояние между точками $M_{i}$ и $M_{j}$.




Доказано [1,2], что всякое бесконечное множество $M \in \mathfrak{M}$ содержится в некоторой прямой $L \subset \mathbb{R}^{2}$
и потому может быть отождествлено с некоторым подмножеством множества $\mathbb{Z}$ целых чисел,
которое, в свою очередь, является объектом изучения теории чисел, но не геометрии.
Таким образом, нас интересуют только конечные множества $M \in \mathfrak{M}$.

\paragraph{Определение.}
Мощностью множества $M$ (обозначается $|M|$), состоящего из конечного
числа элементов, называется количество элементов этого множества.

Для заданного $n \in \mathbb{N}$ обозначим через $\mathfrak{M}_{n}$ множество таких $M \in \mathfrak{M}$,
что $|M|=n$ и $M \not \subset L$ для любой прямой $L \subset \mathbb{R}^{2}$.

Для заданного множество $M \in \mathfrak{M}$ можно вычислить следующие величины,
которые дают важную информацию о нём:
\begin{enumerate}
	\item
		мощность;
	\item
		диаметр;
	\item
		характеристика;
	\item
		минимальное ребро.
\end{enumerate}

\paragraph{Определение.}
Диаметром множества М будем называть наибольшее расстояние между любыми двумя точками множества $M$:
$$
	\operatorname{diam} M=\max_{A, B \in M} |A-B|
	.
$$

\paragraph{Определение.}
Характеристикой множества $M \in \mathfrak{M}_{n}$ называется свободное от квадратов число $р$ такое,
что для любых $A, B, C \in M$ площадь треугольника $ABC$ соизмерима с $\sqrt{p}$.
Пишут: $\operatorname{char}{M}={p}$.

\paragraph{Определение.}
Ребром множества $М$ будем называть любой отрезок $M_{1} M_{2}$, где $M_{1}, M_{2} \in M$.


При исследовании множеств $M\in\mathfrak{M}_n$ часто бывает удобно ввести систему координат.


\paragraph{Теорема о сетке} [3, теорема 4].
Множество $М\in\mathfrak{M}_n$, $\operatorname{char} M = p$ может быть размещено на решётке
$$
	\left\{\left(\frac{a_{i}}{2 m} ; \frac{b_{i} \sqrt{p}}{2 m}\right)\right\}
	,
$$
где $a_{i}, b_{i} \in \mathbb{Z}$, в качестве $m$ можно взять длину любого ребра множества $M$.

Возникает закономерный вопрос о существовании множества $M\in\mathfrak{M}_n$
с заданными характеристикой, диаметром и минимальным ребром.
Известны следующие факты.
\begin{itemize}
	\item
		Для любого $n \in \mathbb{N}, n \geq 3$ выполнено $\mathfrak{M}_{n} \neq \varnothing$ [3, теорема 2].
	\item
		С возрастанием мощности диаметр возрастает не менее чем линейно [4,6].
	\item
		Для любой мощности $n \in \mathbb{N}, n \geq 3$ и любого свободного от квадратов числа $р$
		существует множество $M \in \mathfrak{M}_{n}$ такое, что $\operatorname{char}{M}={p}$ [3, теорема 5].
	\item
		Для любой мощности $n \in \mathbb{N}, n \geq 3$ и любого натурального числа $k$ существует множество
		$M \in \mathfrak{M}_{n}$, содержащее ребро ${k}$~[5].
	\item
		Не существует множества $M \in \mathfrak{M}_{n}$ характеристики 1 с ребром 1 [3].
\end{itemize}
В работе [8] доказано, что
не существует множества $M \in \mathfrak{M}_{3}$ характеристики 2 или 5 с ребром 1,
но существует бесконечные семейства множеств $M_{i} \in \mathfrak{M}_{3}$ с $\operatorname{char} M=3$
и ребром 1 и с $\operatorname{char} M=5$ с ребром $2$.
В данной заметке продолжается исследование множеств $M \in \mathfrak{M}_{3}$ с ребром 1 или 2
и обобщаются результаты [8].
Доказывается, что любое множество $M \in \mathfrak{M}_{3}$ с ребром 1
имеет харакеристику вида $4k+3$, $k\in\mathbb{N}$.
Вместе с тем, для любого свободного от квадратов $p>1$ мы строим бесконечное семейство множеств $M_{i} \in \mathfrak{M}_{3}$,
$\operatorname{char} M=p$ с ребром 2.

Для наглядности рассмотрим следующие частные случаи.

\paragraph{Утверждение 1.}
Не существует множества $M \in \mathfrak{M}_{3}$ характеристики 6  с ребром 1.
\paragraph{Доказательство.}
По теореме о сетке введем систему координат так, что
${O}(0 ; 0), A\left(-\frac{1}{2} ; 0\right), B\left(\frac{1}{2} ; 0\right), C\left(0 ; \frac{b \sqrt{6}}{2}\right)$
и $\{A, B, C\}= M$.
По теореме Пифагора
$$
	\left(\frac{1}{2}\right)^{2}+\left(\frac{b \sqrt{6}}{2}\right)^{2}=r^{2}
	,
$$
$$
	1+6 b^{2}=4 r^{2}\eqno{(1)}
	.
$$
Очевидно, что в уравнении (1) левая часть "--- нечётное число при любом $b \in \mathbb{Z}$,
а правая "--- чётное число при любом $r \in \mathbb{Z}$.
Следовательно, уравнение (1) неразрешимо в целых числах.

\paragraph{Утверждение 2.}
Существует бесконечное семейство множеств $M_{i} \in \mathfrak{M}_{n}$ характеристики 7 с ребром 1.

\paragraph{Доказательство.}
По теореме о сетке введем систему координат так, что
${O}(0 ; 0), A\left(-\frac{1}{2} ; 0\right), B\left(\frac{1}{2} ; 0\right), C\left(0 ; \frac{{b} \sqrt{7}}{2}\right)$ и
и $\{A, B, C\}= M$.
По теореме Пифагора:
$$
	\left(\frac{1}{2}\right)^{2}+\left(\frac{b \sqrt{7}}{2}\right)^{2}=r^{2}
$$
$$
	1+7 b^{2}=4 r^{2}
	.
	\eqno{(2)}
$$
Уравнение (2) имеет решение в целых числах, если $\frac{1+7 b^{2}}{4}$ есть полный квадрат.
Например, $b^{2}=9, {r}^{2}=16$.
Таким образом, $(4 ; 3)$ "--- решение уравнения (2).
Уравнение (2) равносильно следующему уравнению:
$$
	4 r^{2}-7 b^{2}=1
	.
	\eqno{(3)}
$$
Введём замену: $n=2 r$.
Получим
$$
	n^{2}-7 b^{2}=1
	\eqno{(4)}
$$
Уравнение (4) называется уравнением Пелля.

\paragraph{Определение.}
Уравнение Пелля "--- диофантово уравнение вида
$$
x^{2}-m y^{2}=1
,
$$
где $\mathrm{m}$ "--- натуральное число, не являющееся квадратом.
Ясно, что уравнение Пелля имеет тривиальное решение $\mathrm{x}=\pm 1, \mathrm{y}=0$.
\paragraph{Теорема 1} [7].
Любое уравнение Пелля имеет нетривиальное решение.
\paragraph{Теорема 2} [7].
Все нетривиальные положительные решения получаются многократным умножением основного решения на себя.

Применим формулу разности квадратов к уравнению (4) и подставим вместо неизвестных найденное решение:
$$
\begin{array}{l}
(n-b \sqrt{7})(n+b \sqrt{7})=1 \\
(8-3 \sqrt{7})(8+3 \sqrt{7})=1
\end{array}
$$
Согласно теореме 2,
$$
	(8-3 \sqrt{7})^{{k}}(8+3 \sqrt{7})^{{k}}=1^{{k}}, \quad {k} \in \mathbb{N}
	\eqno{(5)}
$$
Заметим, что при возведении в четные степени $k$ получаются нечетные
$n$, и, следовательно, $r$ "--- нецелые числа.
Поэтому будем возводить выражение
(5) в нечётные степени: ${k}=2 m+1$, $m\in\mathbb{N}$.
Приведем несколько решений уравнения (4).

При ${k}=3$:
$$
	(2024-765 \sqrt{7})(2024+765 \sqrt{7})=1
	,
$$
пара чисел
$(2024;765)$ является решением уравнения (4), следовательно, $(1012;765)$ есть решение уравнения (3).

При $k=5$:
$$
	(514088-194307 \sqrt{7})(514088+194307 \sqrt{7})=1
	.
$$
$(514088; 194307)$ "--- решение уравнения (4), $(257044; 194307)$ "--- решение уравнения (3).

Продолжая возводить в нечетные степени обе части уравнения,
получим бесконечную последовательность решений для исходного уравнения (2):
$$
	\left(\frac{x_{n}}{2} ; y_{n}\right), \text { где } x_{n}+y_{n} \sqrt{7}=(8-3 \sqrt{7})^{2 n+1}, n=0,1,2 \ldots
$$
Таким образом, существует бесконечное семейство множеств $M_{i} \in \mathfrak{M}_{n}$ характеристики 7 с ребром 1.

\paragraph{Утверждение 3.}
Всякое множество $M \in \mathfrak{M}_{3}$ с минимальным ребром 1 имеет характеристику вида
$p=4k+3$, $k\in\mathbb{N}$.
\paragraph{Доказательство.}
Пусть минимальное ребро множества $\mathfrak{M}_{3}$ равно 1.
Воспользуемся теоремой о сетке и введем систему координат так, что $О (0;0)$,
$A\left(-\frac{1}{2} ; 0\right)$, $B\left(\frac{1}{2} ; 0\right)$, $C\left(0 ; \frac{b \sqrt{p}}{2}\right)$
и $\{A, B, C\}= M$. По теореме Пифагора:
$$
	\left(\frac{1}{2}\right)^{2}+\left(\frac{{b} \sqrt{{p}}}{2}\right)^{2}={r}^{2}
$$
$$
	1+p b^{2}=4 r^{2}
$$
Для того, чтобы ${r}^{2}$ было целым числом необходимо,
чтобы левая часть уравнения была кратна 4 и ${p}, {b}$ были  нечётными.
Следовательно,  $p b^{2}\equiv3(\operatorname{mod}4)$.
Но $b^{2}\equiv1(\operatorname{mod}4)$,
откуда $p=4 k+3, k\in\mathbb{N}$.

\paragraph{Утверждение 4.}
Для всякого свободного от квадратов числа $p>1$ существует бесконечное семейство множеств $\{M_j\}\subset\mathfrak{M}_3$ с ребром 2
таких, что $\operatorname{char}M_j = p$.
\paragraph{Доказательство.}
По теореме о сетке введем следующую систему
координат:  $O(0 ; 0), A(-1 ; 0), B(1 ; 0), C\left(0 ; \frac{b \sqrt{p}}{4}\right)$ и $A, B, C \in M$.
По теореме Пифагора:
$$
1+\left(\frac{{b} \sqrt{{p}}}{4}\right)^{2}={r}^{2}
$$
Отсюда
$$
1+\frac{{pb}^{2}}{16}={r}^{2}\eqno{(8)}
$$
Правая часть уравнения (8) является целым числом в том случае, если $b=4n$, $n\in\mathbb{N}$.
Тогда
$
1+pn^{2}=r^{2}
$ и
$$
r^{2}-pn^{2}=1\eqno{(9)}
$$
Очевидно, равенство (9) является уравнением Пелля и имеет нетривиальное решение при любом свободном от квадратов числе $p>1$.

\paragraph{Замечание.}
Тот факт, что при $p=1$ уравнение (9) имеет только тривиальное решение,
сам по себе не позволяет опровергнуть существование множеств
$M\in\mathfrak{M}_3$, $\operatorname{char}M=1$ c минимальным ребром 2,
поскольку (в отличие от ситуации с минимальным ребром 1)
искомая третья точка может лежать не только на серединном перпендикуляре к ребру,
но и на гиперболе специального вида.

Список использованной литературы

1. Erdös P. Integral distances // Bull. Amer. Math. Soc. --- 1945. --- Vol. 51, N 12 . ---
P. 996.

2. Anning N. H., Erdös P. Integral distances // Bull. Amer. Math. Soc.-1945.Vol. 51, № 8.-P. 598-600.

3. Авдеев Н.Н., Семенов Е.М. Множества точек с целочисленными расстояниями на плоскости и в Евклидовом пространстве.

4. Solymosi J. Note on integral distances //Discrete \& Computational Geometry. --- 2003. --- T. 30. --- №. 2. --- C. 337-342.

5. Зволинский P. E. Facher integral point sets with particular distances of arbitrary cardinality.
В сборнике: Актуальные проблемы прикладной математики, информатики и механики сборник трудов Международной научной конференции. 2021. С. 668-674

6. Avdeev N. N. On existence of integral point sets and their diameter bounds //Australasian Journal of Combinatorics. – 2020. – Т. 77. – С. 100-116.

7. Бугаенко В. О. Уравнение Пелля. Серия: «Библиотека «Математическое просвещение»».М.: МЦНМО, 2001. --- 32с.: ил.

8. Лушина Е. А. О существовании специальных множеств с целочисленными расстояниями с ребром 1. --- в печати.

On characteristic of planar integral point sets with edges 1 and 2

{\bf Лушина Екатерина Александровна}, студент математического факультета ФГБОУ ВО «Воронежский государственный университет», г.~Воронеж.

{\bf Авдеев Николай Николаевич}, аспирант кафедры теории функций и геометрии математического факультета ФГБОУ ВО «Воронежский государственный университет», г.~Воронеж.

e-mail: nickkolok@mail.ru

Научный руководитель:
{\bf Семенов Евгений Михайлович},
заведующий кафедрой теории функций и геометрии математического факультета ФГБОУ ВО «Воронежский государственный
университет», г. Воронеж.

\end{document}

