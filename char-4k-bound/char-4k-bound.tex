\documentclass[a4paper,14pt]{article} %размер бумаги устанавливаем А4, шрифт 12пунктов
\usepackage[T2A]{fontenc}
\usepackage[utf8]{inputenc}
\usepackage[russian,english]{babel} %используем русский и английский языки с переносами
\usepackage{amssymb,amsfonts,amsmath,mathtext,enumerate,float,amsthm} %подключаем нужные пакеты расширений
\usepackage[unicode,colorlinks=true,citecolor=black,linkcolor=black]{hyperref}
%\usepackage[pdftex,unicode,colorlinks=true,linkcolor=blue]{hyperref}
\usepackage{indentfirst} % включить отступ у первого абзаца
\usepackage[dvips]{graphicx} %хотим вставлять рисунки?
\graphicspath{{illustr/}}%путь к рисункам

\usepackage{geometry} % Меняем поля страницы.
\geometry{left=2cm}% левое поле
\geometry{right=1cm}% правое поле
\geometry{top=2cm}% верхнее поле
\geometry{bottom=2cm}% нижнее поле

\renewcommand{\theenumi}{\arabic{enumi}}% Меняем везде перечисления на цифра.цифра
\renewcommand{\labelenumi}{\arabic{enumi}}% Меняем везде перечисления на цифра.цифра
\renewcommand{\theenumii}{.\arabic{enumii}}% Меняем везде перечисления на цифра.цифра
\renewcommand{\labelenumii}{\arabic{enumi}.\arabic{enumii}.}% Меняем везде перечисления на цифра.цифра
\renewcommand{\theenumiii}{.\arabic{enumiii}}% Меняем везде перечисления на цифра.цифра
\renewcommand{\labelenumiii}{\arabic{enumi}.\arabic{enumii}.\arabic{enumiii}.}% Меняем везде перечисления на цифра.цифра

\sloppy




\renewcommand\normalsize{\fontsize{14}{25.2pt}\selectfont}



\usepackage[backend=biber,style=ajc2020unofficial]{biblatex}
\addbibresource{../common/notmy.bib}
\addbibresource{../common/my.bib}

%\usepackage{cite}

%\usepackage{biblatex2bibitem}

\theoremstyle{plain}
\newtheorem{theorem}{Theorem}[section]
\newtheorem{conjecture}[theorem]{Conjecture}
\newtheorem{lemma}[theorem]{Lemma}
\newtheorem{corollary}[theorem]{Corollary}
\newtheorem{proposition}[theorem]{Proposition}

\theoremstyle{definition}
\newtheorem{definition}[theorem]{Definition}
\newtheorem{construction}[theorem]{Construction}
\newtheorem{remark}[theorem]{Remark}
\newtheorem{problem}[theorem]{Problem}


%Only referenced equations are numbered
\let\etoolboxforlistloop\forlistloop % save the good meaning of \forlistloop
\usepackage{mathtools}
\mathtoolsset{showonlyrefs}
\let\forlistloop\etoolboxforlistloop % restore the good meaning of \forlistloop

% https://tex.stackexchange.com/questions/220268/biblatex-and-autonum-dont-work-together

%\mathtoolsset{showonlyrefs=false}
% (write an equation/multline to be force-numbered here)
%\mathtoolsset{showonlyrefs=true}



\begin{document}

%\renewcommand{\bibname}{Список цитированной литературы}
%\renewcommand\refname{\bibname}
% !!!
% The text starts here

\title{
	On existence of integral point sets and their diameter bounds
	\footnote{
		This work was carried out at Voronezh State University and supported by the Russian Science
		Foundation grant 19-11-00197.
	}
}

\author{
	N.N. Avdeev
	\footnote{nickkolok@mail.ru, avdeev@math.vsu.ru}
	\\
	\textit{Voronezh State University, Voronezh, Russia}
}


\maketitle

\paragraph{Abstract.}
A point set $M$ in $m$-dimensional Euclidean space is called an integral point set if all the distances between the
elements of $M$ are integers, and $M$ is not situated on an $(m-1)$-dimensional hyperplane.
We improve the linear lower bound for diameter of planar integral point sets.
This improvement takes into account some results related to the Point Packing in a Square problem.
Then for arbitrary integers $m \geq 2$, $n \geq m+1$, $d \geq 1$
we give a construction of an integral point set $M$ of $n$ points in $m$-dimensional Euclidean space,
where $M$ contains points $M_1$ and $M_2$ such that the distance between $M_1$ and $M_2$ is exactly $d$.

TODO

\section{Introduction}
Let $\mathbb{N}$ be the set of all positive integers and let $|M_1 M_2|$ denote the Euclidean distance
between points $M_1$ and $M_2$ in a finite-dimensional space $\mathbb{R}^m$
(and, more generally, let $|\Delta|$ denote the length of line segment $\Delta$).
An \textit{integral point set} in $m$-dimensional Euclidean space, $m>2$, is a point set $M$ such that all the distances between the
points of $M$ are integers and $M$ is not situated on an $(m-1)$-dimensional hyperplane.
Erd{\H{o}}s and Anning proved~\cite{anning1945integral,erdos1945integral} that every integral point set consists of a finite number of points.
Taking this into account, we denote the set of all integral point sets of $n$ points in $m$-dimensional Euclidean space by
$\mathfrak{M}(m,n)$ (using the notation in~\cite{our-vmmsh-2018-translit})
and denote the set of all integral point sets in $m$-dimensional Euclidean space by $\mathfrak{M}(m,\mathbb{N})$.
The symbol $\# M$ will be used for the cardinality of $M$, that is the number of points in $M$ in our case.

For every finite point set, its diameter is naturally defined as
\begin{equation}
	\operatorname{diam} M = \max_{A,B\in M} |AB|
	.
\end{equation}
Another emerging question is: how does the diameter of an integral point set depend on its cardinality?
One can easily see that every $M\in\mathfrak{M}(m,n)$ with $\operatorname{diam} M = h$
can be dilated to $M_p\in\mathfrak{M}(m,n)$ with $\operatorname{diam} M = ph$
for every $p\in\mathbb{N}$.
So, the above question should be rephrased:
how does \textit{the least possible} diameter of an integral point set depend on its cardinality?
In order to answer this question, the following function was introduced~\cite{kurz2008bounds,kurz2008minimum}:
\begin{equation}
	d(m,n) = \min_{M\in\mathfrak{M}(m,n)} \operatorname{diam} M = \min_{M\in\mathfrak{M}(m,n)} \max_{A,B\in M} |AB|
	.
\end{equation}
We also refer to~\cite{kurz2008bounds} for a list of known exact values of $d(m,n)$ and its bounds.
In the present paper, we mostly focus on the case $m=2$.

The most significant breakthrough on the planar case was done by Solymosi~\cite{solymosi2003note},
who proved that $cn \leq d(2,n)$ for a sufficiently small constant $c$.
Following Solymosi's proof carefully,
one can derive that the inequality holds at least for $c = 1/24$.
(See~\cite[Exercise 2.6]{garibaldi2005erdos} for some remarks.)
The constant was improved in~\cite{our-mz-rus-translit} to $1/8$ for all $n$ and in~\cite{our-vmmsh-2018-translit}
to $3/8$ for sufficiently large $n$.

The paper~\cite{solymosi2003note} contains one more interesting result.
Let us define a function which is ``dual'' to $d(m,n)$ in some sense:
\begin{equation}
	l(m,n) = \min_{M\in\mathfrak{M}(m,n)} \min_{A,B\in M} |AB|
	.
\end{equation}
Solymosi proved that $l(2,n)\leq 2$.

In the present paper we improve Solymosi's results:
first, we obtain a larger constant $c = 5/11$ in Theorem~\ref{thm:main_estimate},
using the combined approach with the Point Packing in a Square problem
(this approach is different from Solymosi's one);
second, we prove that $l(m,n)=1$ for all possible $m$ and $n$.

Sections 2--4 are devoted to the linear lower bound.
Each section consists of 2 subsections: General results and Special results.
In General results, we do not aim to provide tight estimates;
they are discussed in Special results.

Section 5 is devoted to integral point sets with distance 1 and has its own structure.



\section{Basic Notions and Results}

In this Section, we provide proper definitions and list some known results.

For the sake of brevity, the following designations will be used for numeric sets
of positive integers, non-negative integers and all the integers resp.:
\begin{equation}
	\mathbb{N} = \{1,2,3,4,...\},\quad \mathbb{N}_0 = \mathbb{N} \cup \{0\},
	\quad
	\mathbb{N}_\pm = \{0,\pm 1,\pm 2,\pm 3,\pm 4,...\}
\end{equation}


\begin{lemma}
	\label{lem:triangle_4k_plus_3}
	Any triangle with sides $a \leq b \leq c$, where $c=a+b-1$ and $a, b, c \in \mathbb{N}$,
	has a characteristic of the form $p=4k+3$, $k\in \mathbb{N}_0$.
\end{lemma}

\begin{proof}
	Indeed, let the triangle $ABC$ satisfy: $|BC|=a$, $|AC|=b$. Then, from the triangle inequality $|AB| < |BC|+|AC|$, we can represent the length of side $AB$ as: $|AB|=a+b-s$, $s \in \mathbb{N}$ and $s \leq a$. For $s=1$, the conditions of statement 6 are satisfied: $s$ is an odd number and $s \leq a$.
	Therefore, the triangle with sides $a$, $b$, and $c=a+b-1$ has a characteristic of the form $p=4k+3$, $k\in \mathbb{N}_0$.
\end{proof}




\section{Acknowledgements}
The author thanks Dr. Prof. E.M. Semenov for the fruitful discussion and ideas,
A.S. Chervinskaia for the idea of using the term ``facher'' and proofreading,
Dr.~A.S.~Usachev for proofreading,
and E.A. Momot for the ``SciLexic'' project that helped the author with some word usage issues.
Also the author would like to owe a special thanks to the anonymous referees
for their helpful comments and suggestions on the paper.


\printbibliography


\end{document}
