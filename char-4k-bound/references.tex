\begin{thebibliography}{99}
%
\bibitem{anning1945integral}
N. H. Anning and P. Erdős, Integral distances, \emph{Bull. Amer. Math. Soc.} 51.\textbf{8} (1945), 598–600, \textsc{doi}: \href
{https://doi.org/10.1090/S0002-9904-1945-08407-9} {\nolinkurl {10.1090/S0002-9904-1945-08407-9}}.
%
\bibitem{antonov2008maximal}
A. R. Antonov and S. Kurz, Maximal integral point sets over {$\mathbb {Z}^2$}, \emph{Int. J. Comput. Math.} 87.\textbf{12}
(2008), 2653–2676, \textsc{doi}: \href {https://doi.org/10.1080/00207160902993636} {\nolinkurl {10.1080/00207160902993636}},
arXiv: \href {http://arxiv.org/abs/0804.1280} {\nolinkurl {0804.1280}}.
%
\bibitem{our-vmmsh-2018-translit}
N. N. Avdeev and E. M. Semenov, Integral point sets in the plane and Euclidean space (Множества точек с целочисленными
расстояниями на плоскости и в евклидовом пространстве), \emph{Matematicheskij forum (Itogi nauki. Yug Rossii)} (2018),
217–236.
%
\bibitem{avdeev2019particular}
N. N. Avdeev, R. E. Zvolinsky, and E. A. Momot, On particular diameter bounds for integral point sets in higher dimensions
(2019), arXiv: \href {http://arxiv.org/abs/1909.10386} {\nolinkurl {1909.10386}}, \textsc{url}: \url
{https://ui.adsabs.harvard.edu/abs/2019arXiv190910386A/abstract}.
%
\bibitem{my-semi-general-5-4-bound-2019}
N. Avdeev, On diameter bounds for planar integral point sets in semi-general position (2019), arXiv: \href
{http://arxiv.org/abs/1907.09331} {\nolinkurl {1907.09331}}, \textsc{url}: \url
{https://ui.adsabs.harvard.edu/abs/2019arXiv190709331A/abstract}.%
\bibitem{my-pps-linear-bound-2019}
N. Avdeev, On existence of integral point sets and their diameter bounds, \emph{Australas. J. Combin.} 77.\textbf{1} (2020),
100–116, arXiv: \href {http://arxiv.org/abs/1906.11926} {\nolinkurl {1906.11926}}, \textsc{url}: \url
{https://ui.adsabs.harvard.edu/abs/2019arXiv190611926A/abstract}.
%
\bibitem{costa2013valid}
A. Costa, Valid constraints for the point packing in a square problem, \emph{Discrete Appl. Math.} 161.\textbf{18} (2013),
2901–2909.
%
\bibitem{erdos1945integral}
P. Erdős, Integral distances, \emph{Bull. Amer. Math. Soc.} 51.\textbf{12} (1945), 996, \textsc{doi}: \href
{https://doi.org/10.1090/S0002-9904-1945-08490-0} {\nolinkurl {10.1090/S0002-9904-1945-08490-0}}.
%
\bibitem{harborth1993upper}
H. Harborth, A. Kemnitz, and M. Möller, An upper bound for the minimum diameter of integral point sets, \emph{Discrete
Comput. Geom.} 9.\textbf{4} (1993), 427–432, \textsc{doi}: \href {https://doi.org/10.1007/bf02189331} {\nolinkurl
{10.1007/bf02189331}}.
%
\bibitem{huff1948diophantine}
G. B. Huff, Diophantine problems in geometry and elliptic ternary forms, \emph{Duke Math. J.} 15.\textbf{2} (1948), 443–453,
\textsc{doi}: \href {https://doi.org/10.1215/S0012-7094-48-01543-9} {\nolinkurl {10.1215/S0012-7094-48-01543-9}}.
%
\bibitem{kemnitz1988punktmengen}
A. Kemnitz, Punktmengen mit ganzzahligen Abständen, 1988.
%
\bibitem{kreisel2008heptagon}
T. Kreisel and S. Kurz, There are integral heptagons, no three points on a line, no four on a circle, \emph{Discrete Comput.
Geom.} 39.\textbf{4} (2008), 786–790, \textsc{doi}: \href {https://doi.org/10.1007/s00454-007-9038-6} {\nolinkurl
{10.1007/s00454-007-9038-6}}.
%
\bibitem{kurz2005characteristic}
S. Kurz, On the characteristic of integral point sets in {$\mathbb {E}^m$}, \emph{Australas. J. Combin.} 36 (2006), 241–248,
arXiv: \href {http://arxiv.org/abs/math/0511704} {\nolinkurl {math/0511704}}.
%
\bibitem{kurz2008bounds}
S. Kurz and R. Laue, Bounds for the minimum diameter of integral point sets, \emph{Australas. J. Combin.} 39 (2007), 233–240,
arXiv: \href {http://arxiv.org/abs/0804.1296} {\nolinkurl {0804.1296}}.
%
\bibitem{kurz2008minimum}
S. Kurz and A. Wassermann, On the minimum diameter of plane integral point sets, \emph{Ars Combin.} 101 (2011), 265–287,
arXiv: \href {http://arxiv.org/abs/0804.1307} {\nolinkurl {0804.1307}}.
%
\bibitem{kurz2013constructing}
S. Kurz et al., Constructing $7$-clusters, \emph{Serdica J. Comput.} 8.\textbf{1} (2014), 47–70, arXiv: \href
{http://arxiv.org/abs/1312.2318} {\nolinkurl {1312.2318}}.
%
\bibitem{markot2005newverified}
M. C. Markót and T. Csendes, A new verified optimization technique for the "packing circles in a unit square" problems,
\emph{SIAM J. Optim.} 16.\textbf{1} (2005), 193–219.
%
\bibitem{momot2022example}
E. A. Momot, R. E. Zvolinskiy, and A. E. Zvolinskiy, An example of rails integral point set with cardinality 106, Прикладная
математика и фундаментальная информатика, 2022, 114–115.
%
\bibitem{nozaki2013lower}
H. Nozaki, Lower bounds for the minimum diameter of integral point sets, \emph{Australas. J. Combin.} 56 (2013), 139–143.
%
\bibitem{solymosi2003note}
J. Solymosi, Note on integral distances, \emph{Discrete Comput. Geom.} 30.\textbf{2} (2003), 337–342, \textsc{doi}: \href
{https://doi.org/10.1007/s00454-003-0014-7} {\nolinkurl {10.1007/s00454-003-0014-7}}.
%
\bibitem{solymosi2010question}
J. Solymosi and F. De Zeeuw, On a question of Erdős and Ulam, \emph{Discrete Comput. Geom.} 43.\textbf{2} (2010), 393–401,
arXiv: \href {http://arxiv.org/abs/0806.3095} {\nolinkurl {0806.3095}}.
%
\bibitem{our-ped-2018}
Н. Н. Авдеев, On integral point sets in special position, \emph{Некоторые вопросы анализа, алгебры, геометрии и
математического образования: материалы международной молодежной научной школы «Актуальные направления
математического анализа и смежные вопросы»} 8 (2018), 5–6.%
%
\bibitem{our-pmm-2018}
Н. Н. Авдеев, Об отыскании целоудалённых множеств специального вида, Актуальные проблемы прикладной математики,
информатики и механики - сборник трудов Международной научной конференции. Научно-исследовательские публикации,
2018, 492–498.
%
\bibitem{smurov1998stripcoverings}
М. Смуров and А. Спивак, Покрытия полосками, \emph{Квант} \textbf{5} (1998), 6–12.
\end{thebibliography}
