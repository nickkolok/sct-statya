\documentclass[12pt]{article}
\usepackage[utf8]{inputenc}
\usepackage[T2A]{fontenc}
\usepackage[russian,english]{babel}
\usepackage{amsmath,amsfonts,amssymb,euscript,graphicx,wrapfig,multirow}
\usepackage{dsfont}
\usepackage{amsthm}
\inputencoding{utf8}
%\bibliographystyle{unsrt}
\textheight=240mm \textwidth=170mm
\hoffset=-17mm
\voffset=-17mm


%\usepackage[backend=biber,style=gost-numeric,sorting=none]{biblatex}
%\addbibresource{../common/notmy.bib}
%\addbibresource{../common/my.bib}
%\usepackage{../../../biblatex2bibitem/biblatex2bibitem}

\usepackage{hyperref}
\usepackage{cite}

\makeatletter
%\renewcommand{\fnum@figure}{Figure \thefigure}
\renewcommand{\@biblabel}[1]{#1.}
\makeatother

\theoremstyle{theorem}
\newtheorem{theorem}{Theorem}
\theoremstyle{dfn}
\newtheorem{dfn}{Definition}
\theoremstyle{remark}
\newtheorem{remark}{Remark}

\begin{document}
%\renewcommand\refname{\centering References}


% Переключаем язык на английский.
% Очень полезно как в плане типографики (в том числе расстановки переносов),
% так и в плане того, что не надо переименовывать "Рисунок" в "Figure"
\selectlanguage{english}




\title{
	On particular examples of planar integral point sets and their classification
	\footnote{
		This work was carried out at Voronezh State University and supported by the Russian Science
		Foundation grant 19-11-00197.
	}
}

%% Прекрасно понимаю, что следующая команда - дичь и вордописчество, но время не ждёт, время жмёт
\author{
	Avdeev N.N.
	\footnote{nickkolok@mail.ru, avdeev@math.vsu.ru}
	, Momot E.A., Zvolinsky A.E.
	\\
	\\
	\emph{Voronezh State University}
}

\maketitle

\paragraph{Abstract.}
	A planar integral point set is a set of non-collinear points in plane
	such that for any pair of the  points the Euclidean distance
	between the points is integral.
	We discuss the classification of planar integral point sets
	and provide examples of sets that are not covered by the existent classification.


\section{Introduction}



\begin{dfn}\label{dfn1}
	A planar integral point set (PIPS) is a set $\mathcal{P}$
	of non-collinear points in plane $\mathbb{R}^{2}$ such that
	for any pair of points $P_{1}, P_{2} \in \mathcal{P}$
	the Euclidean distance $|P_{1}P_{2}|$
	between points $P_{1}$ and $P_{2}$ is integral.
\end{dfn}

How do we describe an integral point set?
For example, with the number of points in it, which is always finite~\cite{anning1945integral,erdos1945integral}
and is said to be the \emph{cardinality} of the IPS.
Furthermore, we can naturally % TODO: naturally/essentially? cf. arxiv/1
define the diameter of a finite point set.

\begin{dfn}
	The diameter of the integral point set $\mathcal{P}$ is defined as
	\begin{equation}
		\operatorname{diam(\mathcal{P})} = \underset{P_{1}, P_{2} \in
		\mathcal{P}}{\max} |P_{1}P_{2}|
		.
	\end{equation}
\end{dfn}

Every integral point set also has a characteristic~\cite{kemnitz1988punktmengen,kurz2005characteristic}.
Characteristic of an IPS does not change if the set is moved, dilated ot flipped;
moreover, even addition or deletion of a point does not change the characteristic of IPS.


While the minimal possible diameter for planar integral point sets of given cardinality was being computed,
it was noticed~\cite{kurz2008minimum} that such diameter is attained at sets with many points on a straight line;
for some estimations on this tendency, we also refer the reader to~\cite{solymosi2003note}.
Thus, the following classification was introduced:
\begin{dfn}
	A planar integral point set $M$ is said to be in \emph{semi-general position}
	if no three points of $M$ are located in a straight line.
\end{dfn}

The most dominating examples of PIPS in semi-general position are circular sets.
\begin{dfn}
	A planar integral point sets that is situated on a circle is said to be a \textit{circular}
	point set.
\end{dfn}

So, the following constraint appeared.
\begin{dfn}
	A planar integral point set $M$ is said to be in \emph{general position}
	if no three points of $M$ are located in a straight line
	and no four points of $M$ are located in a circle.
\end{dfn}

Planar integral point sets in general position are very difficult to find;
the first known examples were presented in~\cite{kreisel2008heptagon}.
As for now, there is no known example of PIPS of cardinality 8 in general position.

The main purpose of this work is to provide examples of planar integral point sets
that may give the clue for development of further classification.

For convenience, we use the notation \cite{our-ped-2018,our-pmm-2018,our-vmmsh-2018}:
$\sqrt{p}/q * \{ (x_1,y_1), ...,$ $ (x_n, y_n)  \}$,
which means that each abscissa is multiplied by $1/q$
and each ordinate is multiplied by $\sqrt{p}/q$,  i.e.
\begin{equation}
	\label{eq:char_lattice}
	\sqrt{p}/q * \{ (x_1,y_1), ..., (x_n, y_n)  \}
	=
	\left\{ \left(\frac{x_1}{q},\frac{y_1\sqrt{p}}{q}\right), ..., \left(\frac{x_n}{q},   \frac{y_n\sqrt{p}}{q}\right)  \right\}
	.
\end{equation}
Here $q$ is the characteristic of the PIPS;
every PIPS can be represented in such way~\cite{our-vmmsh-2018-translit}[Theorem 4].

It's notable that any of the examples that are discussed below
is located on a union of at most three straight lines.
%In the most pictures these lines are shown as solid lines.
For classification of planar IPS that are located on a union of two straight lines,
we refer the reader to~\cite{avdeev2019particular}.

There are some examples of planar IPS that are not contained in the union of any three straight lines:
for examples, these are heptagons presented in~\cite{kreisel2008heptagon} and 7-clusters from~\cite{kurz2013constructing}.
However, we have to keep in mind that the circular inversion under certain conditions
translates an integral point set into an integral point set
(although sometimes additional dilation is necessary).
On the other hand, the circular inversion may translate a straight line into a circle and vice versa.
Thus, we can consider \emph{generalized circles}, that are circles or straight lines;
obviously, in that point of view all the examples from~\cite{kreisel2008heptagon} and~\cite{kurz2013constructing}
are located on a union of three generalized circles,
because each seven points are located on a union of a circle and two straight lines.




\section{Rails sets}

\begin{dfn}
	A planar integral point sets of $n$ points with $n-1$ points on a straight line is called
	a \textit{facher} set.
\end{dfn}
Facher sets are very dominating examples of planar integral pont sets.
In~\cite{antonov2008maximal}, facher sets of characteristic 1 are called \textit{semi-crabs}.
%For $9 \leq n \leq 122$, the diameter $d(2,n)$ is reached on a facher point set~\cite{kurz2008minimum}.

\begin{dfn}[\cite{avdeev2019particular}]
	A non-facher planar integral point sets situated in two parallel straight lines
	is called a \textit{rails} set.
\end{dfn}

Among the rails sets, sets with 2 points on one line and all the other on another line dominate.


Two IPSs below have been obtained by dilating~\cite[Fig. 34]{avdeev2019particular} by $23$ and $29$  resp.;
third one has been constructed by dilation and merge.



\begin{figure}[h!]
\center{\includegraphics[width=1\linewidth]{./img/42_symm.png}}
\parbox{1\linewidth}{\caption{IPS of cardinality 42 and diameter 2473117504}
\label{42_symm.png}}
\end{figure}

\begin{figure}[h!]
\center{\includegraphics[width=1\linewidth]{./img/46_symm.png}}
\parbox{1\linewidth}{\caption{IPS of cardinality 46 and diameter 3118278592}
\label{46_symm.png}}
\end{figure}

\begin{figure}[h!]
\center{\includegraphics[width=1\linewidth]{./img/48_symm.png}}
\parbox{1\linewidth}{\caption{IPS of cardinality 48 and diameter 71720407616}
\label{48_symm.png}}
\end{figure}

\begin{itemize}
\setlength{\itemsep}{-1mm}


\item
Figure~\ref{42_symm.png}:
\begin{multline}
	\mathcal{P}_{42}=\sqrt{154}/{1} * \{
		(\pm219513840; -15069600);
		(\pm 345596160; 0);
		(\pm260201760; 0);
		\\
		(\pm225792840; 0);
		(\pm213234840; 0);
		(\pm153961080; 0);
		(\pm144668160; 0);
		(\pm25116000; 0);
		\\
		(\pm694026840; 0);
		(\pm514710560; 0);
		(\pm359116940; 0);
		(\pm13423904; 0);
		(\pm75682880; 0);
		\\
		(\pm464143680; 0);
		(\pm827069880; 0);
		(\pm92144325; 0);
		(\pm1195180740; 0);
		(\pm1236558752; 0);
		\\
		(\pm44590560; 0);
		(\pm339925740; 0);
		(\pm117312468; 0)
	\}
\end{multline}

\item
Figure~\ref{46_symm.png}:
\begin{multline}
	\mathcal{P}_{46}=
	\sqrt{154}/{1} * \{
		(\pm276778320; -19000800);
		(\pm435751680; 0);
		(\pm328080480; 0);
		\\
		(\pm268861320; 0);
		(\pm194124840; 0);
		(\pm182407680; 0);
		(\pm1559139296; 0);
		(\pm284695320; 0);
		\\
		(\pm1506967020; 0);
		(\pm1042827240; 0);
		(\pm875077320; 0);
		(\pm648982880; 0);
		(\pm585224640; 0);
		\\
		(\pm452799620; 0);
		(\pm95426240; 0);
		(\pm16925792; 0);
		(\pm116181975; 0);
		(\pm428602020; 0);
		\\
		(\pm56222880; 0);
		(\pm769560480; 0);
		(\pm626458560; 0);
		(\pm31668000; 0);
		(\pm130761918; 0)
	\}
\end{multline}

\item
Figure~\ref{48_symm.png}:
\begin{multline}
	\mathcal{P}_{48}=
	\sqrt{154}/{1} * \{
		( \pm6365901360 ; -437018400);
		( \pm10022288640; 0);
		( \pm23985026520 ; 0);
		\\
		( \pm389293216 ; 0);
		( \pm6183810360 ; 0);
		( \pm4464871320 ; 0);
		( \pm4195376640 ; 0);
		( \pm728364000 ; 0);
		\\
		( \pm35860203808 ; 0);
		( \pm34660241460 ; 0);
		( \pm7545851040 ; 0);
		( \pm20126778360 ; 0);
		\\
		( \pm14926606240 ; 0);
		( \pm13460166720 ; 0);
		( \pm10414391260 ; 0);
		( \pm2194803520 ; 0);
		\\
		( \pm2672185425 ; 0);
		( \pm9857846460 ; 0);
		( \pm1293126240 ; 0);
		( \pm17699891040 ; 0);
		\\
		( \pm14408546880 ; 0);
		( \pm3007524114 ; 0);
		( \pm6547992360 ; 0);
		( \pm3402061572 ; 0)
	\}
\end{multline}

\end{itemize}

Taking the examples into consideration,
we can conjecture that there is an infinite point set with rational distances
that contains $\mathcal{P}_{48}$.
(However, it is known~\cite{solymosi2010question} that
if a point set $S$ with rational distances has infinitely many points on a line or on a circle,
then all but 4 resp. 3 points of $S$ are on the line or on the circle.)


\section{Example of sets with many common points that cannot be merged}

Figure~\ref{8_with_many_common} shows an example of three PIPS of cardinality 8,
each pair of that shares 6 or 7 points but cannot be united into another PIPS.

\begin{figure}[h!]
	\begin{minipage}[h]{0.32\linewidth}
		\begin{center}
			\includegraphics[width=1\linewidth]{./img/8_2520_143_symm1.png}\\ a)
		\end{center}
	\end{minipage}
	\hfill
	\begin{minipage}[h]{0.32\linewidth}
		\begin{center}
			\includegraphics[width=1\linewidth]{./img/8_2520_143_symm2.png}\\ b)
		\end{center}
	\end{minipage}
	\begin{minipage}[h]{0.32\linewidth}
		\begin{center}
			\includegraphics[width=1\linewidth]{./img/8_2520_143_4a27_other.png}\\ c)
		\end{center}
	\end{minipage}
	\hfill
	\caption{IPSs with cardinality 8 and diameter 2520 with many common points}
	\label{8_with_many_common}
\end{figure}

\begin{itemize}
\item
$\mathcal{P}=\sqrt{143}/2*\{
( \pm1620 ; 0);
( \pm1920 ; 300);
( \pm735 ; 75);
( \pm340 ; 0);
\}$

\item
$\mathcal{P}=\sqrt{143}/2*\{
( \pm1620 ; 0);
( \pm1920 ; 300);
( \pm735 ; 75);
( \pm1767 ; 147);
\}$

\item
$\mathcal{P}=
\sqrt{143}/2*\{
( \pm1620 ; 0);
( \pm1920 ; 300);
( \pm735 ; 75);
( -340 ; 0);
( 1767 ; 147);
\}$

\end{itemize}

The distance between the non-adoptable points is
\begin{equation}
	\sqrt{\left(\frac{1767}{2} - \frac{340}{2}\right)^2 + \left(\frac{147}{2}\right)^2\cdot143} = 2\sqrt{320401}
	.
\end{equation}
It's notable that 320401 is a prime number.

\section{Integral point sets with two axes of symmetry}

\begin{figure}[h!]
\center{\includegraphics[width=1\linewidth]{./img/15_61960444_1_other.png}}
\parbox{1\linewidth}{\caption{IPS of cardinality 15 and diameter 61960444}
\label{15_61960444_1.png}}
\end{figure}

\begin{figure}[h!]
\center{\includegraphics[width=1\linewidth]{./img/15_45970652_1_other.png}}
\parbox{1\linewidth}{\caption{IPS of cardinality 15 and diameter 45970652}
\label{15_45970652_1.png}}
\end{figure}

\begin{itemize}
\item
$\mathcal{P}=\sqrt{1}/1*\{
( \pm9406950 ; 0);
( \pm7063350 ; 0);
( 2962050 ; \pm3437280);
( 0 ; 0);
( -2962050 ; \pm3437280);\\
( \pm2188662 ; 0);
( \pm30980222 ; 0);
( \pm2914854 ; 0);
\}$

\item
$\mathcal{P}=\sqrt{1}/1*\{
( \pm6979350 ; 0);
( \pm5240550 ; 0);
( 2197650 ; \pm2550240);
( 0 ; 0);
( -2197650 ; \pm2550240);\\
( \pm1623846 ; 0);
( \pm22985326 ; 0);
( \pm2476152 ; 0);
\}$
\end{itemize}

%TODO:

%TODO: 11_14450_1_8693fad981c82dc92d7c9365bcb9945e

%The set on Fig. TODO has been obtained from the set on Fig. TODO by dilation by 5.

%The set on Fig. TODO has been obtained from the set on Fig. TODO by dilation by 42 (or less? TODO).

%\section{Arrow-like integral point sets with one axis of symmetry}

%TODO:

%на каждое такое множество: исходный вариант, все этапы растращаивания с коэффициентом растаращивания и пояснением, кто из кого.


\section{Other examples}

Figure~\ref{8_2535_1_d680} displays an IPS with
no axis of symmetry;
although the set is of characteristic 1,
it cannot be extended by reflecting relatively to the $x$ axis.
Moreover, we failed to extend it by dilation and looking for extra points on the $x$ axis.
%
%TODO: write about points which prevent the set from reflection
%
\begin{multline}
	\mathcal{P}_8=
	\sqrt{1}/13*
	\{
	( 0 ; 0);
	( 8450 ; 0);
	( 12844 ; 0);
	( 21294 ; 0);
	( 29575 ; 0);
	\\
	( -2366 ; -8112);
	( 10647 ; -14196);
	( 15022 ; -3696)
	\}
\end{multline}
%
\begin{figure}[h!]
\center{\includegraphics[width=1\linewidth]{./img/8_2535_1_d680ff2d172ebe40250a54ca9419af34_other.png}}
\parbox{1\linewidth}{\caption{IPS of cardinality 8 and diameter 2535}
\label{8_2535_1_d680}}
\end{figure}

The set shown on Figure~\ref{8_2400_42_56f3} has an axis of symmetry, but it is $y$ axis, not $x$ axis
and due to the fact that its characteristic is not 1,
the set cannot be rotated by $90^\circ$ but still stay on lattice~\eqref{eq:char_lattice}:
\begin{equation}
	\mathcal{P}_{8y}=
	\sqrt{42}/1*\{( \pm1200 ; 0);
	( \pm529 ; 182);
	( \pm814 ; 152);
	( \pm440 ; 80)
	\}
\end{equation}

\begin{figure}[h!]
	\center{\includegraphics[width=1\linewidth]{./img/8_2400_42_56f33d8dc5df552595e807c438bc0f55_other.png}}
	\parbox{1\linewidth}{\caption{IPS of cardinality 8 and diameter 2400}
	\label{8_2400_42_56f3}}
\end{figure}



\section{Final remarks}
All the given planar integral point sets were obtained through a combination of computer search and intuition of the authors.

The source code can be obtained at https://gitlab.com/Nickkolok/ips-algo

%\printbibliography
%\printbibitembibliography

\begin{thebibliography}{99}
%
\bibitem{anning1945integral}
\emph{Anning} \emph{N. H.}, \emph{Erdős} \emph{P.}. Integral distances //
Bulletin of the American Mathematical Society. — 1945. — Vol. 51, no. 8. —
Pp. 598–600. — DOI: \href {https://doi.org/10.1090/S0002-9904-1945-08407-9}
{\nolinkurl {10.1090/S0002-9904-1945-08407-9}}.
%
\bibitem{erdos1945integral}
\emph{Erdős} \emph{P.}. Integral distances // Bulletin of the American Mathematical
Society. — 1945. — Vol. 51, no. 12. — P. 996. — DOI: \href
{https://doi.org/10.1090/S0002-9904-1945-08490-0} {\nolinkurl
{10.1090/S0002-9904-1945-08490-0}}.%
\bibitem{kemnitz1988punktmengen}
\emph{Kemnitz} \emph{A.}. Punktmengen mit ganzzahligen Abständen. — 1988.
%
\bibitem{kurz2005characteristic}
\emph{Kurz} \emph{S.}. On the characteristic of integral point sets in {$\mathbb
{E}^m$} // Australasian Journal of Combinatorics. — 2006. — Vol. 36. —
Pp. 241–248. — arXiv: \href {http://arxiv.org/abs/math/0511704} {\nolinkurl
{math/0511704}}.
%
\bibitem{kurz2008minimum}
\emph{Kurz} \emph{S.}, \emph{Wassermann} \emph{A.}. On the minimum diameter
of plane integral point sets // Ars Combinatoria. — 2011. — Vol. 101. — Pp. 265–287. —
arXiv: \href {http://arxiv.org/abs/0804.1307} {\nolinkurl {0804.1307}}.
%
\bibitem{solymosi2003note}
\emph{Solymosi} \emph{J.}. Note on integral distances // Discrete \& Computational
Geometry. — 2003. — Vol. 30, no. 2. — Pp. 337–342. — DOI: \href
{https://doi.org/10.1007/s00454-003-0014-7} {\nolinkurl {10.1007/s00454-003-0014-7}}.
%
\bibitem{kreisel2008heptagon}
\emph{Kreisel} \emph{T.}, \emph{Kurz} \emph{S.}. There are integral heptagons, no
three points on a line, no four on a circle // Discrete \& Computational Geometry. —
2008. — Vol. 39, no. 4. — Pp. 786–790. — DOI: \href
{https://doi.org/10.1007/s00454-007-9038-6} {\nolinkurl {10.1007/s00454-007-9038-6}}.
%
\bibitem{our-ped-2018}
\emph{Авдеев} \emph{Н. Н.}. On integral point sets in special position // Некоторые
вопросы анализа, алгебры, геометрии и математического образования: материалы
международной молодежной научной школы «Актуальные направления
математического анализа и смежные вопросы». — 2018. — Vol. 8. — Pp. 5–6.
%
\bibitem{our-pmm-2018}
\emph{Авдеев} \emph{Н. Н.}. Об отыскании целоудалённых множеств
специального вида // Актуальные проблемы прикладной математики,
информатики и механики - сборник трудов Международной научной
конференции. — Научно-исследовательские публикации, 2018. — Pp. 492–498.
%
\bibitem{our-vmmsh-2018}
\emph{Авдеев} \emph{Н. Н.}, \emph{Семёнов} \emph{Е. М.}. Множества точек с
целочисленными расстояниями на плоскости и в евклидовом пространстве //
Математический форум (Итоги науки. Юг России). — 2018. — Pp. 217–236.%
\bibitem{our-vmmsh-2018-translit}
\emph{Avdeev} \emph{N. N.}, \emph{Semenov} \emph{E. M.}. Integral point sets in
the plane and Euclidean space (Множества точек с целочисленными расстояниями
на плоскости и в евклидовом пространстве) // Matematicheskij forum (Itogi nauki.
Yug Rossii). — 2018. — Pp. 217–236.
%
\bibitem{avdeev2019particular}
\emph{Avdeev} \emph{N. N.}, \emph{Zvolinsky} \emph{R. E.},
\emph{Momot} \emph{E. A.}. On particular diameter bounds for integral point sets in
higher dimensions. — 2019. — arXiv: \href {http://arxiv.org/abs/1909.10386}
{\nolinkurl {1909.10386}}. — URL: \url
{https://ui.adsabs.harvard.edu/abs/2019arXiv190910386A/abstract}.
%
\bibitem{kurz2013constructing}
Constructing $7$-clusters / S. Kurz [et al.] // Serdica Journal of Computing. — 2014. —
Vol. 8, no. 1. — Pp. 47–70. — arXiv: \href {http://arxiv.org/abs/1312.2318} {\nolinkurl
{1312.2318}}.
%
\bibitem{antonov2008maximal}
\emph{Antonov} \emph{A. R.}, \emph{Kurz} \emph{S.}. Maximal integral point sets
over {$\mathbb {Z}^2$} // International Journal of Computer Mathematics. —
2008. — Vol. 87, no. 12. — Pp. 2653–2676. — DOI: \href
{https://doi.org/10.1080/00207160902993636} {\nolinkurl
{10.1080/00207160902993636}}. — arXiv: \href {http://arxiv.org/abs/0804.1280}
{\nolinkurl {0804.1280}}.
%
\bibitem{solymosi2010question}
\emph{Solymosi} \emph{J.}, \emph{De Zeeuw} \emph{F.}. On a question of Erdős
and Ulam // Discrete \& Computational Geometry. — 2010. — Vol. 43, no. 2. —
Pp. 393–401. — arXiv: \href {http://arxiv.org/abs/0806.3095} {\nolinkurl
{0806.3095}}.
\end{thebibliography}

\end{document}
