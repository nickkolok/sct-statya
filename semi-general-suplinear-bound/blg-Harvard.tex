\begin{engbibliography}{99}

\bibitem{ENGanning1945integral}
	Anning, N.H. \& Erdös, P. 1945.
	“Integral distances”.
	\emph{Bulletin of the American Mathematical Society}, vol. 51.8, pp. 598—600.
	doi: 10.1090/S0002-9904-1945-08407-9

\bibitem{ENGerdos1945integral}
	Erdös, P. 1945.
	“Integral distances”.
	\emph{Bulletin of the American Mathematical Society}, vol. 51.12, p. 996.
	doi: 10.1090/S0002-9904-1945-08490-0

\bibitem{ENGour-vmmsh-2018}
	Avdeev, N.N. \& Semenov, E.M. 2018.
	``Mnozhestva tochek c tselochislennymi rasstoyaniyami na ploskosti i v evklidovom prostranstve''
	(``Integral point sets on the plane and in Euclidean space'')
	\emph{Matematicheskiy forum (Itogi nauki. Yug Rossii)}, pp. 217—236.
	% No DOI

\bibitem{ENGkurz2008bounds}
	Kurz, S. \& Laue, R. 2007.
	“Bounds for the minimum diameter of integral point sets”.
	\emph{Australasian Journal of Combinatorics}, vol. 39, pp. 233—240. arXiv: 0804.1296.

\bibitem{ENGkurz2008minimum}
	Kurz, S. \& Wassermann, A. 2011.
	“On the minimum diameter of plane integral point sets”.
	\emph{Ars Combinatoria}, vol. 101, pp. 265—287. arXiv: 0804.1307.

\bibitem{ENGantonov2008maximal}
	Antonov, A.R. \& Kurz, S. 2008.
	“Maximal integral point sets over $Z^2$”.
	\emph{International Journal of Computer Mathematics}, vol. 87.12, pp. 2653—2676. arXiv: 0804.1280.
	doi: 10.1080/00207160902993636

\bibitem{ENGhuff1948diophantine}
	Huff, G.B. 1948.
	“Diophantine problems in geometry and elliptic ternary forms”.
	\emph{Duke Mathematical Journal}, vol. 15.2, pp. 443—453.
	doi:10.1215/S0012-7094-48-01543-9

\bibitem{ENGharborth1993upper}
	Harborth, H., Kemnitz, A. \& Möller, M. 1993.
	“An upper bound for the minimum diameter of integral point sets”.
	\emph{Discrete \& Computational Geometry} vol. 9.4, pp. 427—432.
	doi: 10.1007/bf02189331

\bibitem{ENGpiepmeyer1996maximum}
	Piepmeyer, L. 1996.
	“The maximum number of odd integral distances between points in the plane”.
	\emph{Discrete \& Computational Geometry}, vol. 16.1, pp. 113—115.
	doi: 10.1007/bf02711135

\bibitem{ENGkreisel2008heptagon}
	Kreisel, T. \& Kurz, S. 2008.
	“There are integral heptagons, no three points on a line, no four on a circle”.
	\emph{Discrete \& Computational Geometry}, vol. 39.4, pp. 786—790.
	doi: 10.1007/s00454-007-9038-6

\bibitem{ENGkurz2013constructing}
	Kurz, S., Noll, L.C., Rathbun, R, \& Simmons, C. 2014.
	“Constructing 7-clusters”.
	\emph{Serdica Journal of Computing}, vol. 8.1, pp. 47—70. arXiv: 1312.2318.

\bibitem{ENGsolymosi2003note}
	Solymosi, J. 2003.
	“Note on integral distances”.
	\emph{Discrete \& Computational Geometry}, vol. 30.2, pp. 337—342.
	doi: 10.1007/s00454-003-0014-7

\bibitem{ENGmy-pps-linear-bound-2019}
	Avdeev, N. 2019.
	“On existence of integral point sets and their diameter bounds”.
	arXiv: 1906.11926

\bibitem{ENGbat2018number}
	Bat-Ochir, G. 2018.
	“On the number of points with pairwise integral distances on a circle”.
	\emph{Discrete Applied Mathematics}, vol. 254, pp. 17—32.
	doi: 10.1016/j.dam.2018.07.004

\bibitem{ENGbrass2006research}
	Brass, P., Moser, W.O.J. \& Pach, J. 2006.
	\emph{Research problems in discrete geometry}. Springer Science \& Business Media.
	doi: 10.1007/0-387-29929-7

\bibitem{ENGguy2013unsolved}
	Guy, R. 2013.
	\emph{Unsolved problems in number theory}. Vol. 1.
	Springer Science \& Business Media.
	doi: 10.1007/978-1-4757-1738-9

\bibitem{ENGour-pmm-2018}
	Avdeev, N.N. 2018.
	“Ob otyskanii tseloudalennykh mnozhestv spetsial'nogo vida”
	(``On the search of integral point sets of a special type'').
	\emph{
		Aktual'nye problemy prikladnoj matematiki, informatiki i mekhaniki
		- sbornik trudov Mezhdunarodnoj nauchnoj konferencii.
	}
	(Actual problems of applied mathematics, informatics and mechanics
	- proc. of the int. conf.).
	Voronezh, pp. 492—498.

\bibitem{ENGour-ped-2018}
	Avdeev, N.N. 2018.
	“On integral point sets in special position”.
	\emph{Nekotorye voprosy analiza, algebry, geometrii i matematicheskogo obrazovaniya}
	(Some problems of analysis, algebra, geometry and mathematical education), vol. 8, pp. 5—6.

\bibitem{ENGkurz2005characteristic}
	Kurz, S. 2006.
	“On the characteristic of integral point sets in $E^m$”.
	\emph{Australasian Journal of Combinatorics}, vol. 36, pp. 241–248.
	arXiv: math/0511704.

\bibitem{ENGnozaki2013lower}
	Nozaki, H. 2013.
	“Lower bounds for the minimum diameter of integral point sets”.
	\emph{Australasian Journal of Combinatorics}, vol. 56, pp. 139—143.

\bibitem{ENGsmurov1998stripcoverings}
	Smurov, M. \& Spivak, A. 1998.
	“Pokrytiya poloskami” (``Covering by strips'').
	\emph{Kvant}, vol. 5, pp. 6--12

\end{engbibliography}
