\documentclass[11pt,twoside,draft
]{article}
\usepackage{amsmath,amsfonts,amssymb,amsthm,indentfirst,enumerate,textcomp}
\usepackage[utf8]{inputenc}
\usepackage[T2A]{fontenc}
\usepackage{chebsb}
\usepackage[english,russian]{babel}
\usepackage{indentfirst, array}
\usepackage{amscd,latexsym}
\usepackage{mathrsfs}
%\usepackage[ruled, linesnumbered]{algorithm2e}
\usepackage{tabularx}
\usepackage{multirow}

\usepackage{graphicx}
\usepackage{textcase}% Оформление страниц




%%%%%%%%%%%%%%%%%%%%%%%%%%%%%%%%%%





\label{beg}
% заглавие стати и аннотация

\levkolonttl{%левый колонтитул - авторы
N.~N.~Avdeev}
\prvkolonttl{%правый колонтитул - сокращенное название статьи
On diameter bounds for planar integral point sets \ldots}

\UDK{% УДК статьи
519.146}

\DOI{
10.22405/2226-8383-2018-\tom-\iss-\pageref{beg}-\pageref{end}
}

\title
{%название статьи на русском языке с указанием источника финансирования при необходимости
Об оценках на диаметр плоских множеств с целочисленными расстояниями полуобщего положения%
\footnote{Исследование выполнено за счет гранта Российского научного фонда (проект 19-11-00197).}}
{%название статьи на английском языке
On diameter bounds for planar integral point sets in semi-general position}

\author
{%авторы статьи с указанием города на русском языке
Н. Н. Авдеев (г. Воронеж)}
{%авторы статьи с указанием города на английском языке
N. N. Avdeev (Voronezh)}

\Cite{
Н. Н. Авдеев. Об оценках на диаметр плоских множеств с целочисленными расстояниями полуобщего положения // Чебышевcкий сборник, 2019, т.~\tom, вып.~\iss, с.~\pageref{beg}--\pageref{end}.
}
{N. N. Avdeev, 2019, "On diameter bounds for planar integral point sets in semi-general position"\,, {\it Che\-by\-shev\-skii sbornik}, vol.~\tom, no.~\iss, pp.~\pageref{beg}--\pageref{end}.
}

\info
{%авторы статьи с указанием аффилиации на русском языке
\noindent {\bf Н. Н. Авдеев}~--- Воронежский государственный университет, кафедра теории функций и геометрии

\noindent
\emph{e-mail: nickkolok@mail.ru, avdeev@math.vsu.ru}


}
{%авторы статьи с указанием аффилиации на английском языке
\noindent {\bf N. N. Avdeev}~--- Voronezh State University

\noindent
\emph{e-mail: nickkolok@mail.ru, avdeev@math.vsu.ru}

}

\Abstract
{%Аннотация статьи на русском языке 150-250 слов с учетом ключевых слов
Множество точек $M$ на плоскости называется плоским множеством с целочисленными расстояниями,
если все расстояния между точками $M$ суть целые числа,
и при этом $M$ не содержится ни в какой прямой.
Говорят, что плоское множество с целочисленными расстояниями есть множество полуобщего положения,
если никакие три его точки не лежат на одной прямой.
Известная оценка снизу на диаметр плоского множества с целочисленными расстояниями
линейна относительно его мощности.
Ранее не были известны отдельные оценки снизу на диаметр плоских множеств с целочисленными расстояниями полуобщего положения
заданной мощности
(известная конструктивная оценка сверху на диаметр плоских множеств с целочисленными расстояниями
использует как раз множества полуобщего положения).
В статье доказывается надлинейная оценка снизу
на диаметр плоского множества с целочисленными расстояниями полуобщего положения
(полиномиальная с показателем $5/4$).
Доказательство основано на относительно большом количестве лемм и наблюдений,
включая результаты Солимоси
из статьи, в которой была впервые доказана линейная оценка снизу
на диаметр плоских множеств с целочисленными расстояниями.
Полученная в данной статье оценка всё ещё не точна.
}
{%Аннотация статьи на английском языке
A point set $M$ in the Euclidean plane is said to be a planar integral point set if all the distances between the
elements of $M$ are integers, and $M$ is not situated on a straight line.
A planar integral point set is said to be a set in semi-general position, if it does not contain collinear triples.
The existing lower bound for mininal diameter of a planar integral point set is linear with respect to its cardinality.
There were no known special diameter bounds for planar integral point sets in semi-general position of given cardinality
(the known upper bound for planar integral point sets is constructive
and employs planar integral point sets in semi-general position).
We prove a new lower bound for minimal diameter of planar integral point sets in semi-general position
that is better than linear (polynomial of power $5/4$).
The proof is based on several lemmas and observations, including the ones established by Solymosi
to prove the first linear lower bound for diameter of a planar integral point set.
We have to admit that our bound is still not exact.
}

\keywords
{%ключевые слова на русском языке
комбинаторная геометрия, диаметр множества, множество с целочисленными расстояниями.
}
{%ключевые слова на английском языке
combinatorial geometry, diameter of a set, integral point set.
}

%число наименований в библиографии
\Bibliography{21 название.}{21 titles.}

\begin{document}

%генерация заглавия статьи
\maketitle

\enmaketitle



\section{Введение}



\textit{Множеством с целочисленными расстояниями (МЦР)} на плоскости
называется такое множество точек $M$, что расстояние (евклидово) между любыми двумя точками из $M$
есть целое число, и при этом $M$ не является подмножеством никакой прямой.
Каждое МЦР состоит лишьиз конечного числа точек~\cite{anning1945integral,erdos1945integral};
поэтому мы можем обозначить множество всех множеств из $n$ точек с целочисленными расстояниями через
$\mathfrak{M}(2,n)$ (придерживаясь обозначений~\cite{our-vmmsh-2018} и в целом для краткости изложения)
и естественным образом определить диаметр множества $M\in\mathfrak{M}(2,n)$:
\begin{equation}
	\operatorname{diam} M = \max_{A,B\in M} |AB|
	,
\end{equation}
где $|AB|$ обозначает обычное (евклидово) расстояние между точками $A$ и $B$.
Через $\# M$ будем обозначать мощность множества $M$, которая в нашем случае всегда совпадает с количеством точек в $M$.

Поскольку любое МЦР можно очевидным образом растянуть
до множества большего диаметра,
интерес представляет в первую очередь минимальный возможный диаметр при заданной мощности множества.
Это рассуждение формализуется следующей функцией, введённой в~\cite{kurz2008bounds,kurz2008minimum}:
\begin{equation}
	d(2,n) = \min_{M\in\mathfrak{M}(2,n)} \operatorname{diam} M
	.
\end{equation}

Выяснилось, что достаточно просто сконструировать МЦР на плоскости,
все точки которого, кроме одной, лежат на некоторой прямой
(такие множества называются \textit{веерными});
аналогично --- для 2 точек вне прямой (см., напр., ~\cite{antonov2008maximal}, где такие множества названы \textit{крабовыми})
и даже для 4 точек вне прямой~\cite{huff1948diophantine}.
Для $9\leq n\leq 122$ минимальный возможный диаметр МЦР мощности $n$ достигаетя именно на веерном множестве~\cite{kurz2008bounds}.


\begin{definition}
	Множество $M\in\mathfrak{M}(2,n)$ есть МЦР \textit{полуобщего положения},
	если никакие три точки из $M$ не лежат на одной прямой.
	Кратко будем писать, что $M\in \overline{\mathfrak{M}}(2,n)$.
\end{definition}

В~\cite{harborth1993upper} приводится конструкция МЦР полуобщего положения для любой наперёд заданной мощности;
получаемое МЦР содержится в некоторой окружности.
Кроме того, в~\cite{piepmeyer1996maximum} приведена изящная конструкция МЦР $M$ любой наперёд заданной мощности,
содержащегося в окружности,
для любого возможного количества нечётных расстояний между точками $M$.

\begin{definition}
	Множество $M\in\overline{\mathfrak{M}}(2,n)$ есть МЦР \textit{общего положения},
	если никакие четыре точки из $M$ не лежат на одной окружности.
	Кратко будем писать, что $M\in \dot{\mathfrak{M}}(2,n)$.
\end{definition}

Остаётся неясным, любую ли мощность может имет МЦР общего положения;
однако, известны~\cite{kreisel2008heptagon,kurz2013constructing} примеры МЦР $M\in \dot{\mathfrak{M}}(2,7)$.

Очевидно неравенство
\begin{equation*}
	d(2,n) \leq \overline{d}(2,n) \leq \dot{d}(2,n)
	,
\end{equation*}
где
$
	\overline{d}(2,n) = \min_{M\in\overline{\mathfrak{M}}(2,n)} \operatorname{diam} M
$
и
$
	\dot{d}(2,n) = \min_{M\in\dot{\mathfrak{M}}(2,n)} \operatorname{diam} M
$.
Выполнено, однако, и более интересное соотношение:
\begin{equation*}
	c_1 n \leq d(2,n) \leq \overline{d}(2,n) \leq n^{c_2 \log \log n}
	.
\end{equation*}
Верхняя оценка установлена в~\cite{harborth1993upper}.
Нижняя оценка впервые была получена в~\cite{solymosi2003note};
наибольшее известное значение для $c_1$ равно $5/11$ (для $n\geq 4$)~\cite{my-pps-linear-bound-2019}.

Известны некоторые оценки на диаметр МЦР специального вида.
В частности, для множеств, в которых достаточно много точек лежит на одной прямой,
доказана следующая теорема.
\begin{theorem}~\cite[Theorem 4]{kurz2008minimum}
	Для любых $\delta > 0$, $\varepsilon > 0$ и множества $P\in\mathfrak{M}(2,n)$,
	в котором не менее $n^\delta$ лежат на некоторой прямой, существует число $n_0 (\varepsilon)$
	такое, что для всех $n \geq n_0 (\varepsilon)$ верно неравенство
	\begin{equation}
		\operatorname{diam} P \geq n^{\frac{\delta}{4 \log 2(1+\varepsilon)}\log \log n}
		.
	\end{equation}
\end{theorem}
Об оценках на диаметр МЦР,
содержащихся в окружности, можно прочесть в~\cite{bat2018number}.

Отдельные случаи МЦР на плоскости также обсуждались
в~\cite[\S 5.11]{brass2006research},~\cite[\S D20]{guy2013unsolved},~\cite{our-pmm-2018},~\cite{our-ped-2018}.
За обобщением на пространства более высоких размерностей и соответствующими оценками
отсылаем читателя к~\cite{kurz2005characteristic,nozaki2013lower}.

В настоящей статье мы доказываем отдельную оценку на диаметр множества с целочисленными расстояниями полуобщего положения.
Требование полуобщего положения является существенным для доказательства.



\section{Вспомогательные утверждения}

В этом параграфе приводятся определения и леммы, которые в дальнейшем потребуются
для доказательства основного результата.


\begin{lemm}
	\cite[Наблюдение 1]{solymosi2003note}
	Пусть стороны треугольник $T$ есть целые числа $a \leq b \leq c$,
	тогда наименьшая из высот $m$ этого треугольника не менее $\left(a - \frac{1}{4}\right)^{1/2}$.
\end{lemm}

\begin{definition}
	Полосой ширины $\rho$ будем называть часть плоскости,
	заключённую между двумя параллельными прямыми,
	отстоящими друг от друга на расстояние $\rho$.
\end{definition}

\begin{lemm}
	\cite{smurov1998stripcoverings}
	Пусть треугольник $T$, наименьшая из высот которого равна $\rho$, расположен в полосе.
	Тогда ширина этой полосы не менее $\rho$.
\end{lemm}

\begin{corollary}
	\label{cor:solymosi_strip}
	Пусть треугольник $T$ с целыми сторонами $a \leq b \leq c$ лежит в некоторой полосе,
	тогда ширина этой полосы не менее $\left(a - \frac{1}{4}\right)^{1/2}$.
\end{corollary}


\begin{lemm}
	\cite[Lemma 4]{our-vmmsh-2018};
	\cite[Lemma 2.4]{my-pps-linear-bound-2019}
	\label{lem:square_container}
	Пусть $M\in\mathfrak{M}(2,n)$ и $\operatorname{diam} M = d$.
	Тогда $M$ расположено в некотором квадрате со стороной $d$.
\end{lemm}

\begin{definition}
	\cite[Определение 2.5]{my-pps-linear-bound-2019}
	Будем называть \textit{крестом} точек $M_1$ и $M_2$ и обозначать через $cr(M_1,M_2)$ объединение двух прямых:
	прямой, проходящей через точки $M_1$ и $M_2$,
	и серединного перпендикуляра к отрезку $M_1 M_2$.
\end{definition}

\begin{lemm}
	\cite[Theorem 3.10]{my-pps-linear-bound-2019}
	\label{lem:no_distance_one}
	Каждое множество $M\in\mathfrak{M}(2,n)$,
	для некоторых точек которого $M_1,M_2 \in M$ выполнено равенство $|M_1 M_2|=1$,
	состоит из $n-1$ точек, включая $M_1$ и $M_2$, лежащих на некоторой прямой $m$,
	и ещё одной точки, лежащей вне $m$ на серединном перпендикуляре к отрезку $M_1 M_2$.
\end{lemm}


\begin{lemm}
	\label{lem:count_of_points_on_hyperbolas}
	Пусть $\{M_1, M_2, M_3, M_4\} \subset M\in\overline{\mathfrak{M}}(2,n)$
	(точки $M_2$ и $M_3$ могут совпадать, остальные попарно различны), $n\geq 4$.
	Тогда $\# M \leq 4 \cdot |M_1 M_2| \cdot |M_3 M_4|$.
\end{lemm}

\begin{remark}
	Лемма~\ref{lem:count_of_points_on_hyperbolas} явлется вариацией~\cite{erdos1945integral}.
\end{remark}

\begin{proof}[Доказательство]
	Для каждой точки $N\in M$ выполнено одно из следующих условий:

	a) $N$ принадлежит $cr(M_1,M_2)$~--- итого не более 4 точек (не более 2 на каждой из прямых);

	b) $N$ принадлежит $cr(M_3,M_4)$~--- итого не более 4 точек (не более 2 на каждой из прямых);

	c) $N$ принадлежит пересечению одной из $|M_1 M_2| - 1$ гипербол
	с одной из $|M_3 M_4| - 1$ гипербол~--- итого не более  $4 (|M_1 M_2| - 1)(|M_3 M_4| - 1)$ точек;

	По лемме~\ref{lem:no_distance_one} $|M_1 M_2| \geq 2$ и $|M_3 M_4| \geq 2$.
	Из того, что
	\begin{multline}
		4 (|M_1 M_2| - 1)(|M_3 M_4| - 1) + 4 + 4
		=
		4 ( (|M_1 M_2| - 1)(|M_3 M_4| - 1) + 2)
		=
		\\=
		4 ( |M_1 M_2| \cdot |M_3 M_4| + 1 - |M_1 M_2| - |M_3 M_4| + 2)
		=
		\\=
		4 ( |M_1 M_2| \cdot |M_3 M_4| + (1 - |M_1 M_2|) + (2 - |M_3 M_4|))
		\leq
		4 |M_1 M_2| \cdot |M_3 M_4|
		,
	\end{multline}
	и следует утверждение леммы.
\end{proof}


\section{Основной результат}

\begin{theorem}
	\label{thm:main_result}
	Для всякого целого $n \geq 3$ выполнено неравенство
	\begin{equation}
		\overline{d}(2,n) \geq (n/5)^{5/4}
		.
	\end{equation}
\end{theorem}

\begin{proof}[Доказательство]
	Для $n = 3$ имеем $\overline{d}(2,3) = 1$ (достигается на равностороннем треугольнике со стороной 1),
	и утверждение теоремы очевидно.
	Рассмотрим $M\in\overline{\mathfrak{M}}(2,n)$, $n \geq 4$, $\operatorname{diam} M = p$.

	Выберем точки $M_1, M_2, M_3, M_4 \in M$
	(точки $M_2$ и $M_3$ могут совпадать, остальные должны быть попарно различны) таким образом, что
	\begin{equation}
		\min_{A, B \in M} |AB| = |M_1 M_2|
		,
	\end{equation}
	\begin{equation}
		\min_{A, B \in M \setminus \{M_1\}} |AB| = |M_3 M_4| = m
		.
	\end{equation}

	Для $m \leq p^{2/5}$ по лемме~\ref{lem:count_of_points_on_hyperbolas}
	\begin{equation}
		n \leq 4 \cdot |M_1 M_2| \cdot |M_3 M_4| \leq  4 p^{4/5}
		.
	\end{equation}
	Или, что то же самое,
	\begin{equation}
		\label{eq:hyperbolas_5_4}
		p \geq (n/4) ^ {5/4} > (n/5) ^ {5/4}
		.
	\end{equation}

	Рассмотрим теперь $m > p^{2/5}$.
	Тогда для любых точек $A,B \in M\setminus\{M_1\}$ выполнено неравенство $|AB| > p^{2/5}$.
	Согласно следствию~\ref{cor:solymosi_strip}, никакие три точки из  $M\setminus\{M_1\}$
	не лежат в полосе шириной $p^{1/5} / 2$.

%TODO: Solymosi's proof accepts a=b; in our case, a=b leads to a better bound (?)
% and the condition a < b may (??) remove -1/4 from the lemma about triangle


	В силу леммы~\ref{lem:square_container} множетво $M$ расположено в квадрате со стороной $p$.
	Покроем этот квадрат объединением $q$ полос, $2p^{4/5} \leq q < 2p^{4/5} + 1$ так,
	что ширина каждой полосы не превышает $p^{1/5} / 2$.
	Каждая из полученных полос содержит не более двух точек из  $M\setminus\{M_1\}$,
	поэтому
	\begin{equation}
		\label{eq:strips_4_5}
		n \leq 2(2p^{4/5} + 1) + 1
		= 4p^{4/5}+3
		\leq 5 p^{4/5}
		.
	\end{equation}
	Последнее неравенство выполнено в силу того, что $\overline{d}(2,n) \geq 4$ для $n\geq 4$~~\cite{kurz2008minimum}
	% TODO: add two more links from the kurz2008minimum's intro in case of lack of references
	и $4^{4/5}>3$.
	Из неравенства~\eqref{eq:strips_4_5} легко получаем, что
	\begin{equation}
		\label{eq:strips_5_4}
		p \geq (n/5) ^ {5/4}
		.
	\end{equation}
\end{proof}



\begin{remark}
	Известен следующий факт~\cite[Следствие 1]{solymosi2003note}:
\end{remark}

\begin{lemm}
	Пусть $H \in \overline {\mathfrak{M}}(2,n)$.
	Тогда минимальное расстояние между точками $H$ не менее $n^{1/3}$.
\end{lemm}
Применяя аналогичные приёмы, можно получить, что
\begin{equation}
	n \leq 3 \frac{\operatorname{diam} H }{n^{1/6}}
	,
\end{equation}
откуда выводится оценка
\begin{equation}
	\overline{d}(2,n) \geq c_3 n^{7/6}
	,
\end{equation}
которая, очевидно, слабее оценки в теореме~\ref{thm:main_result}.


\section{Заключение}
Полученная в данной статье оценка ---
первая специальная оценка диаметра снизу для множества с целочисленными расстояниями полуобщего положения.
Поэтому автор не преследовал цели получить как можно б\'{о}льшую константу в теореме~\ref{thm:main_result},
предпочитая сохранить ясность и прозрачность доказательства.
Очевидно, что в результате более тщательного исследования эта константа может быть увеличена.
Кроме того, и верхние, и нижние оценки по-прежнему, увы, остаются далеки от точных.

\section{Благодарности}
Автор благодарит проф. Е.М. Семёнова за вычитку и статьи и ценные советы;
к.ф.-м.н. А.С. Усачёва за вычитку английского варианта статьи.


%библиография по ГОСТу

% Мне библиографию в требуемом формате сгенерировал biblatex:
% https://github.com/odomanov/biblatex-gost/issues/20

\begin{thebibliography}{99}
\bibitem{anning1945integral}
Anning N. H., Erdös P. Integral distances // Bulletin of the American Mathematical
Society. — 1945. — т. 51, No 8. — с. 598—600.
\bibitem{erdos1945integral}
Erdös P. Integral distances // Bulletin of the American Mathematical Society. — 1945. —
т. 51, No 12. — с. 996.
\bibitem{our-vmmsh-2018}
Авдеев Н. Н., Семёнов Е. М. Множества точек с целочисленными расстояниями на
плоскости и в евклидовом пространстве // Математический форум (Итоги науки. Юг
России). — 2018. — с. 217—236.
\bibitem{kurz2008bounds}
Kurz S., Laue R. Bounds for the minimum diameter of integral point sets // Australasian
Journal of Combinatorics. — 2007. — т. 39. — с. 233—240. — arXiv: 0804.1296.
\bibitem{kurz2008minimum}
Kurz S., Wassermann A. On the minimum diameter of plane integral point sets // Ars
Combinatoria. — 2011. — т. 101. — с. 265—287. — arXiv: 0804.1307.
\bibitem{antonov2008maximal}
Antonov A. R., Kurz S. Maximal integral point sets over $\mathbb{Z}^2$ // International Journal of
Computer Mathematics. — 2008. — т. 87, No 12. — с. 2653—2676. — arXiv: 0804.1280.
\bibitem{huff1948diophantine}
Huff G. B. Diophantine problems in geometry and elliptic ternary forms // Duke
Mathematical Journal. — 1948. — т. 15, No 2. — с. 443—453.
\bibitem{harborth1993upper}
Harborth H., Kemnitz A., Möller M. An upper bound for the minimum diameter of integral
point sets // Discrete \& Computational Geometry. — 1993. — т. 9, No 4. — с. 427—432.
\bibitem{piepmeyer1996maximum}
Piepmeyer L. The maximum number of odd integral distances between points in the plane //
Discrete \& Computational Geometry. — 1996. — т. 16, No 1. — с. 113—115.
\bibitem{kreisel2008heptagon}
Kreisel T., Kurz S. There are integral heptagons, no three points on a line, no four on a
circle // Discrete \& Computational Geometry. — 2008. — т. 39, No 4. — с. 786—790.
\bibitem{kurz2013constructing}
Constructing 7-clusters / S. Kurz [и др.] // Serdica Journal of Computing. — 2014. — т. 8,
No 1. — с. 47—70. — arXiv: 1312.2318.
\bibitem{solymosi2003note}
Solymosi J. Note on integral distances // Discrete \& Computational Geometry. — 2003. —
т. 30, No 2. — с. 337—342.
\bibitem{my-pps-linear-bound-2019}
Avdeev N. On existence of integral point sets and their diameter bounds. //
Australasian Journal of Combinatorics. — 2020. — т. 77.1. — с. 100—116. — arXiv: 1906.11926.
\bibitem{bat2018number}
Bat-Ochir G. On the number of points with pairwise integral distances on a circle // Discrete
Applied Mathematics. — 2018. — т. 254. — с. 17—32.
\bibitem{brass2006research}
Brass P., Moser W. O., Pach J. Research problems in discrete geometry. — Springer Science
\& Business Media, 2006.
\bibitem{guy2013unsolved}
Guy R. Unsolved problems in number theory. т. 1. — Springer Science \& Business Media,
2013.
\bibitem{our-pmm-2018}
Авдеев Н. Н. Об отыскании целоудалённых множеств специального вида // Актуальные
проблемы прикладной математики, информатики и механики - сборник трудов Международной
научной конференции. — Научно-исследовательские публикации, 2018. — с. 492—498.
\bibitem{our-ped-2018}
Авдеев Н. Н. On integral point sets in special position // Некоторые вопросы анализа,
алгебры, геометрии и математического образования: материалы международной молодежной
научной школы «Актуальные направления математического анализа и смежные вопросы». —
2018. — т. 8. — с. 5—6.
\bibitem{kurz2005characteristic}
Kurz S. On the characteristic of integral point sets in $E^m$ // Australasian Journal of
Combinatorics. — 2006. — т. 36. — с. 241. — arXiv: math/0511704.
\bibitem{nozaki2013lower}
Nozaki H. Lower bounds for the minimum diameter of integral point sets // Australasian
Journal of Combinatorics. — 2013. — т. 56. — с. 139—143.
\bibitem{smurov1998stripcoverings}
Смуров М., Спивак А. Покрытия полосками // Квант. — 1998. — No 5. — с. 6.
\end{thebibliography}




%библиография по Гарвардскому стандарту

\begin{engbibliography}{99}

\bibitem{ENGanning1945integral}
	Anning, N.H. \& Erdös, P. 1945.
	“Integral distances”.
	\emph{Bulletin of the American Mathematical Society}, vol. 51.8, pp. 598—600.
	doi: 10.1090/S0002-9904-1945-08407-9

\bibitem{ENGerdos1945integral}
	Erdös, P. 1945.
	“Integral distances”.
	\emph{Bulletin of the American Mathematical Society}, vol. 51.12, p. 996.
	doi: 10.1090/S0002-9904-1945-08490-0

\bibitem{ENGour-vmmsh-2018}
	Avdeev, N.N. \& Semenov, E.M. 2018.
	``Mnozhestva tochek c tselochislennymi rasstoyaniyami na ploskosti i v evklidovom prostranstve''
	(``Integral point sets on the plane and in Euclidean space'')
	\emph{Matematicheskiy forum (Itogi nauki. Yug Rossii)}, pp. 217—236.
	% No DOI

\bibitem{ENGkurz2008bounds}
	Kurz, S. \& Laue, R. 2007.
	“Bounds for the minimum diameter of integral point sets”.
	\emph{Australasian Journal of Combinatorics}, vol. 39, pp. 233—240. arXiv: 0804.1296.

\bibitem{ENGkurz2008minimum}
	Kurz, S. \& Wassermann, A. 2011.
	“On the minimum diameter of plane integral point sets”.
	\emph{Ars Combinatoria}, vol. 101, pp. 265—287. arXiv: 0804.1307.

\bibitem{ENGantonov2008maximal}
	Antonov, A.R. \& Kurz, S. 2008.
	“Maximal integral point sets over $Z^2$”.
	\emph{International Journal of Computer Mathematics}, vol. 87.12, pp. 2653—2676. arXiv: 0804.1280.
	doi: 10.1080/00207160902993636

\bibitem{ENGhuff1948diophantine}
	Huff, G.B. 1948.
	“Diophantine problems in geometry and elliptic ternary forms”.
	\emph{Duke Mathematical Journal}, vol. 15.2, pp. 443—453.
	doi:10.1215/S0012-7094-48-01543-9

\bibitem{ENGharborth1993upper}
	Harborth, H., Kemnitz, A. \& Möller, M. 1993.
	“An upper bound for the minimum diameter of integral point sets”.
	\emph{Discrete \& Computational Geometry} vol. 9.4, pp. 427—432.
	doi: 10.1007/bf02189331

\bibitem{ENGpiepmeyer1996maximum}
	Piepmeyer, L. 1996.
	“The maximum number of odd integral distances between points in the plane”.
	\emph{Discrete \& Computational Geometry}, vol. 16.1, pp. 113—115.
	doi: 10.1007/bf02711135

\bibitem{ENGkreisel2008heptagon}
	Kreisel, T. \& Kurz, S. 2008.
	“There are integral heptagons, no three points on a line, no four on a circle”.
	\emph{Discrete \& Computational Geometry}, vol. 39.4, pp. 786—790.
	doi: 10.1007/s00454-007-9038-6

\bibitem{ENGkurz2013constructing}
	Kurz, S., Noll, L.C., Rathbun, R, \& Simmons, C. 2014.
	“Constructing 7-clusters”.
	\emph{Serdica Journal of Computing}, vol. 8.1, pp. 47—70. arXiv: 1312.2318.

\bibitem{ENGsolymosi2003note}
	Solymosi, J. 2003.
	“Note on integral distances”.
	\emph{Discrete \& Computational Geometry}, vol. 30.2, pp. 337—342.
	doi: 10.1007/s00454-003-0014-7

\bibitem{ENGmy-pps-linear-bound-2019}
	Avdeev, N. N. 2019.
	“On existence of integral point sets and their diameter bounds”.
	\emph{Australasian Journal of Combinatorics}, vol. 77.1, pp. 100—116.
	arXiv: 1906.11926

\bibitem{ENGbat2018number}
	Bat-Ochir, G. 2018.
	“On the number of points with pairwise integral distances on a circle”.
	\emph{Discrete Applied Mathematics}, vol. 254, pp. 17—32.
	doi: 10.1016/j.dam.2018.07.004

\bibitem{ENGbrass2006research}
	Brass, P., Moser, W.O.J. \& Pach, J. 2006.
	\emph{Research problems in discrete geometry}. Springer Science \& Business Media.
	doi: 10.1007/0-387-29929-7

\bibitem{ENGguy2013unsolved}
	Guy, R. 2013.
	\emph{Unsolved problems in number theory}. Vol. 1.
	Springer Science \& Business Media.
	doi: 10.1007/978-1-4757-1738-9

\bibitem{ENGour-pmm-2018}
	Avdeev, N.N. 2018.
	“Ob otyskanii tseloudalennykh mnozhestv spetsial'nogo vida”
	(``On the search of integral point sets of a special type'').
	\emph{
		Aktual'nye problemy prikladnoj matematiki, informatiki i mekhaniki
		- sbornik trudov Mezhdunarodnoj nauchnoj konferencii.
	}
	(Actual problems of applied mathematics, informatics and mechanics
	- proc. of the int. conf.).
	Voronezh, pp. 492—498.

\bibitem{ENGour-ped-2018}
	Avdeev, N.N. 2018.
	“On integral point sets in special position”.
	\emph{Nekotorye voprosy analiza, algebry, geometrii i matematicheskogo obrazovaniya}
	(Some problems of analysis, algebra, geometry and mathematical education), vol. 8, pp. 5—6.

\bibitem{ENGkurz2005characteristic}
	Kurz, S. 2006.
	“On the characteristic of integral point sets in $E^m$”.
	\emph{Australasian Journal of Combinatorics}, vol. 36, pp. 241–248.
	arXiv: math/0511704.

\bibitem{ENGnozaki2013lower}
	Nozaki, H. 2013.
	“Lower bounds for the minimum diameter of integral point sets”.
	\emph{Australasian Journal of Combinatorics}, vol. 56, pp. 139—143.

\bibitem{ENGsmurov1998stripcoverings}
	Smurov, M. \& Spivak, A. 1998.
	“Pokrytiya poloskami” (``Covering by strips'').
	\emph{Kvant}, vol. 5, pp. 6--12

\end{engbibliography}




\label{end}

\end{document}
