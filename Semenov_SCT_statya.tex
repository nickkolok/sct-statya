\documentclass[a4paper,14pt]{article} %размер бумаги устанавливаем А4, шрифт 12пунктов
\usepackage[T2A]{fontenc}
\usepackage[utf8]{inputenc}
\usepackage[english,russian]{babel} %используем русский и английский языки с переносами
\usepackage{amssymb,amsfonts,amsmath,mathtext,cite,enumerate,float,amsthm} %подключаем нужные пакеты расширений
\usepackage[pdftex,unicode,colorlinks=true,linkcolor=blue]{hyperref}
\usepackage{indentfirst} % включить отступ у первого абзаца
\usepackage[dvips]{graphicx} %хотим вставлять рисунки?
\graphicspath{{illustr/}}%путь к рисункам

\makeatletter
\renewcommand{\@biblabel}[1]{#1.} % Заменяем библиографию с квадратных скобок на точку:
\makeatother %Смысл этих трёх строчек мне непонятен, но поверим "Запискам дебианщика"

\usepackage{geometry} % Меняем поля страницы. 
\geometry{left=1cm}% левое поле
\geometry{right=1cm}% правое поле
\geometry{top=1cm}% верхнее поле
\geometry{bottom=2cm}% нижнее поле

\renewcommand{\theenumi}{\arabic{enumi}}% Меняем везде перечисления на цифра.цифра
\renewcommand{\labelenumi}{\arabic{enumi}}% Меняем везде перечисления на цифра.цифра
\renewcommand{\theenumii}{.\arabic{enumii}}% Меняем везде перечисления на цифра.цифра
\renewcommand{\labelenumii}{\arabic{enumi}.\arabic{enumii}.}% Меняем везде перечисления на цифра.цифра
\renewcommand{\theenumiii}{.\arabic{enumiii}}% Меняем везде перечисления на цифра.цифра
\renewcommand{\labelenumiii}{\arabic{enumi}.\arabic{enumii}.\arabic{enumiii}.}% Меняем везде перечисления на цифра.цифра

\begin{document}\large


УДК 511.95 + 514.112

\begin{center}
	\textbf{\LARGE О множествах точек на плоскости с целочисленными расстояниями}
\end{center}

{\Large{Н.Н. Авдеев, Е.М. Семёнов.}}


\paragraph{\S 1} % Махровейшее вордописчество =(
Известна следующая

\paragraph{Теорема 1.} Пусть $\{M_1,M_2\}$ --- счётное множество точек на плоскости и расстояние $|M_i,M_j|\in \mathbb{N}$ для всех $1\leq i < j <\infty$, где $\mathbb{N}$ --- множество натуральных чисел. Тогда найдется такая прямая на плоскости $l$, что $M_i\in l$ для всех $i \in \mathbb{N}$.

Формулировка теоремы и идея её доказательства приведены в \cite{Newman}, problem 29. Полное доказательство можно найти в \cite{angem1kurs}.
Там же показано, что для любого $n\in \mathbb{N}$ существует такое множество $\{M_1,M_2,...,M_n\}\subset \mathbb{R}^2$, что $|M_i,M_j|$ для всех $1\leq i < j  \leq n$ и $M_1, M_2, ..., M_n$ не лежат на прямой.
Изучению таких подмножеств посвящена настоящая работа.


Для заданного $n\in \mathbb{N}$ обозначим через $C_n$ множество таких последовательностей $M_1,M_2,...,M_n \in \mathbb{R}^2$, что $|M_i,M_j|\in\mathbb{N}$ для всех $1\leq i < j  \leq n$ и  $M_1,M_2,...,M_n$ не принадлежат никакой прямой.
Положим
$$
F(n)=\min\limits_{A\in C_n} d(A),
$$
где $d(A)$ --- диаметр $A$, т. е.
$$
d(A)=\max\limits_{x,y\in A}|x,y|
$$

Точную асимптотику последовательности  $F(n)$ найти не удалось, получены лишь верхняя и нижняя оценки.

\paragraph{\S 2}
Число элементов множества $A$ обозначим через $|A|$.
В \cite{angem1kurs} была доказана

\paragraph{Лемма 2.}

Пусть $A=(M_1, M_2, ..., M_n) \in C_n$ для некоторого $n\in N$ и $M_1, M_2, M_3$ не принадлежат прямой.
Тогда $n\leq (a+1)(b+1)+3$, где $a=|M_1,M_2|$, $|M_2,M_3|$.

Аналогичное утверждение справедливо, когда $M_1,M_2,M_3$ принадлежат некоторой прямой и $M_2$ лежит между $M_1$ и $M_3$.
В этом случае
$n\leq (a+1)(b+1)+3+d(A)$.



\paragraph{Лемма 3.}
Пусть $m\in\mathbb{N}, m\geq 4$, последовательность $(M_1,M_2,...,M_{2m^2+1})$ принадлежит $C_{2m^2+1}$ и содержится в квадрате со стороной $d$.
Тогда $d>\frac{1}{2}m^2$.

\paragraph{Доказательство.} Разобьём квадрат со стороной $d$ на $m^2$ квадратов со стороной $\frac{d}{m}$.
Тогда по крайней мере один из маленьких квадратов содержит некоторые три точки исходной последовательности.
Без ограничения общности $M_1,M_2,M_3$ содержатся в квадрате со стороной $\frac{d}{m}$.
Поэтому $|M_1,M_2|,|M_2,M_3| \leq \frac{d}{m}\sqrt{2}$.
В силу леммы 2
$$
2m^2+1 \leq\left(\frac{d}{m}\sqrt{2}+1\right)^2+3+d\sqrt{2}
$$

Положим $d=\lambda m^2$.
Тогда
$$
2m^2+1 \leq \left(\lambda m \sqrt2 + 1 \right)^2 + 3 + \lambda m^2 \sqrt2
$$
и
$$
0\leq \left( 2\lambda^2 + \sqrt2 \lambda - 2 \right)m^2 + 2\sqrt2 \lambda m + 3
$$

Для $m\geq 4$ это неравенство не выполнено, если $\lambda \leq \frac{1}{2}$.
Поэтому $\lambda > \frac{1}{2}$ и $d>\frac{m^2}{2}$.
$\Box$


Обозначим через $p_i, i\in\mathbb{N}$ простые числа, начиная с 3.
По теореме Чебышева (\cite{Buhshtab}, теорема 325) $p_i \leq bi\ln(i+1)$ для некоторого $b>0$ и всех $i\in\mathbb{N}$.
Обозначим
$$
A_n=\prod_{i=1}^{n} p_i, n\in\mathbb{N}
$$

Тогда
$$
A_n\leq b^n n! \prod_{i=1}^{n} \ln(i+1)
$$

и по формуле Стирлинга
\begin{equation}\label{ocenka_Stirling}
A_n <
b^n \sqrt{2\pi n} \left(\frac{n}{e}\right)^n \left(1+\frac{1}{n}\right) \prod_{i=1}^{n} \ln(i+1)
\leq \left(\frac{bn\ln (n+1)}{e}\right)^n
\end{equation}


\paragraph{Теорема 4.}
$F(2^n)<A_n$ для всех $n\in\mathbb{N}$.

\paragraph{Доказательство.}
Через $S$ обозначим множество подмножеств $\{1,2,..,n\}$ и каждому $I\in S$ поставим в соответствие числа $c_I=\prod\limits_{c\in I}p_i, b_I=\frac{1}{2}\left(c_I-\frac{A_n}{c_I}\right)$.
Так как $c_I$ и $\frac{A_n}{c_I}$ нечётны, то $b_I$ --- целые числа.

Рассмотрим подмножество точек на плоскости
\begin{equation}\label{konstrukcia_primorialy}
M_I=\{(b_I,0), I\in S\}, N=(0, \sqrt{A_n})
\end{equation}


Если $I,J \in S, I \neq J$, то $M_I \neq M_J$.
Поэтому множество (\ref{konstrukcia_primorialy}) содержит $2^n+2$ элементов.
Так как
\begin{equation*}
	|M_I,N|=\left(\frac{1}{4}\left( c_I^2 - 2 A_n + \frac{A_n^2}{c_I^2} \right)+A_n\right)^\frac{1}{2}=
	\left(\frac{1}{4}\left( c_I^2 - 2 A_n + \frac{A_n^2}{c_I^2} \right)+A_n\right)^\frac{1}{2}=
	\frac{c_I+\frac{A_n}{C_I}}{2} \in \mathbb{N},
\end{equation*}
то все расстояния между точками множества (\ref{konstrukcia_primorialy}) есть целые числа.
Диаметр множества (\ref{konstrukcia_primorialy}) достигается на паре точек $M_{(1,2,...,n)}$ и $M_{\varnothing}$, для которых
$$
|M_{(1,2,...,n)},M_{\varnothing}|=\frac{1}{2}(A_n-1)-\frac{1}{2}(1-A_n) = A_n-1 < A_n
$$

Поэтому 
$$
F(2^n+2) < A_n
$$

Отсюда, из \ref{ocenka_Stirling} и монотонности $F(n)$ вытекает, что
\begin{equation}\label{ocenka_primoryaly}
F(n) \leq \left( \frac{b \log_2(n+1) \ln \log_2 (n+1)}{e}\right)^{\log_2(n+1)}
\end{equation}
для всех $n\in \mathbb{N}$.


Заметим, что подобная конструкция не позволяет получить оценку сверху в виде полинома.
Это вызвано тем, что количество точек в множестве подобного типа (подмножество прямой и точка, на ней не лежащая) ограничено числом  делителей числа $2\mu^2+2$ (эту оценку обосновать?), где $\mu$ --- расстояние от точки до прямой.
С другой стороны, в \cite{Ramanujan} (формулы 198-200) показано, что
$$
p(n)=\max\limits_{1\leq k \leq n} d(k),
$$
где $d(k)$ --- количество делителей числа $k$ (дивизор-функция Рамануджана?),
растёт медленнее любого полинома.

\paragraph{Теорема 5.}

$$
 \max\left( \frac{n-5}{8} \right) \leq F(n) \leq \left( \frac{b \log_2(n+1) \ln \log_2 (n+1)}{e}\right)^{\log_2(n+1)}
$$
для всех $n\in\mathbb{N}$.

\paragraph{Доказательство.}
Верхняя оценка непосредственно следует из \ref{ocenka_primoryaly}.

Пусть $\left(M_1, M_2, ..., M_n\right) \in C_n$.
Найдём такое $m\in \mathbb{N}$, что
$$
2m^2+1 \leq n < 2(m+1)^2 +1
$$

Пусть $d$ --- сторона минимального квадрата, содержащего все точки $M_i, 1\leq i \leq m^2+1$.
Используя монотонность последовательности $F(n)$ и лемму 3, получаем для $n\geq 17$
$$
F(n) \leq F\left(2m^2+1\right)\geq d \geq \frac{1}{2}m^2 > \frac{1}{2}\left(\left(\frac{n-1}{2}\right)^\frac{1}{2} - 1 \right)
$$

Простые оценки показывают, что
$$
\frac{1}{2}\left(\left(\frac{n-1}{2}\right)^\frac{1}{2} -1 \right) > \frac{n-5}{8}
$$

Отсюда вытекает, что
$$
\frac{n-5}{8}<F(n)
$$

Для $3 \leq n < 15$ значения $F(n)$ были вычислены на ЭВМ (см. далее).
$\Box$

В ходе численного расчёта были вычислены значения $F(n)$ для $4\leq n \leq 42$.

\begin{table}[H]
\caption{Результаты численного расчёта}
\label{tabular:pc_counted}
\begin{center}
\begin{tabular}{|c|c|c|c|c|c|c|c|c|c|c|c|}
\hline
\textbf{n}    &  3 & 4 & 5 & 6 &  7 &  8 &  9 & 10 & 11 & 12 & 13 \\
\hline
\textbf{F(n)} &  1 & 4 & 7 & 8 & 17 & 21 & 29 & 40 & 51 & 63 & 74  \\
\hline
\hline
\textbf{n}    &  14 &  15 &  16 &  17 &  18 &  19 &  20 &  21 &  22 &  23 &  24 \\
\hline
\textbf{F(n)} &  91 & 104 & 121 & 134 & 153 & 164 & 196 & 212 & 234 & 256 & 286  \\
\hline
\hline
\textbf{n}    &  25 &  26 &  27 &  28 &  29 &  30 &  31 &  32 &  33 &  34 &  35 \\
\hline
\textbf{F(n)} & 304 & 338 & 370 & 384 & 414 & 448 & 464 & 494 & 524 & 553 & 578  \\
\hline
\end{tabular}
\begin{tabular}{|c|c|c|c|c|c|c|c|c|}
\hline
\textbf{n}    &  36 &  37 &  38 &  39 &  40 &  41 &  42 &  43  \\
\hline
\textbf{F(n)} & 608 & 642 & 667 & 692 & 754 & 816 & 897 & >956  \\
\hline
\end{tabular}
\end{center}
\end{table}

Заметим, что соответствующие множества нельзя расположить на целочисленной решётке, но можно расположить на решётке 
$$
	\left(
		\left\{\pm\frac{\sqrt{p_1}}{q_1}\right\};
		\left\{\pm\frac{\sqrt{p_2}}{q_2}\right\}
	\right),$$
где $p_1,q_1,p_2,q_2 \in \mathbb{N} \cup{0}$.

(Надо ли это доказывать?).

В большинстве обсчитанных случаев оптимальное расположение точек определяется однозначно с точностью до симметрий.
Тем не менее, например, для $n=18$ имеем два набора:

\begin{multline*}
\{( 0 ; 0 ); ( 0 ; 153 ); ( 0 ; 144 ); ( 0 ; 130 ); ( 0 ; 115 ); ( 0 ; 111 ); ( 0 ; 104 ); ( 0 ; 98 ); ( 0 ; 88 ); ( 0 ; 76 );\\
( 0 ; 66 ); ( 0 ; 60 ); ( 0 ; 53 ); ( 0 ; 49 ); ( 0 ; 34 ); ( 0 ; 20 ); ( 0 ; 11 ); ( \sqrt{2880} ; 82 )\}
\end{multline*}
и
\begin{multline*}
\{( 0 ; 0 ); ( 0 ; 153 ); ( 0 ; 134 ); ( 0 ; 121 ); ( 0 ; 104 ); ( 0 ; 98 ); ( 0 ; 93 ); ( 0 ; 85 ); ( 0 ; 76 ); ( 0 ; 69 );\\
( 0 ; 65 ); ( 0 ; 58 ); ( 0 ; 49 ); ( 0 ; 41 ); ( 0 ; 36 ); ( 0 ; 30 ); ( 0 ; 13 ); ( \sqrt{1440} ; 67 )\}
\end{multline*}


Также примечательно, что, начиная с $n=9$, для всех обсчитанных $n$ оптимальное множество, кроме одной точки, лежит на прямой (т. е. $n-1$ точек на прямой и одна точка вне её).

(Я готов выложить текст программы, которой считал. Нужно ли дать ссылку?)

Время, требуемое для вычисления $F(n)$ разработанными алгоритмами, растёт, как установлено эмпирически, не медленнее, чем $n^4$.
Так, расчёт для $n=5$ занимает меньше секунды, а для вычисления $F(41)$ при известном (вычисленном ранее) $F(40)$ ушло больше суток.

Авторы благодарят проф. Ю. А. Брудного за информацию о работе \cite{Newman} и ценные замечания.

Работа выполнена при поддержке РНФ, грант ???.


\addcontentsline{toc}{chapter}{Литература}
\begin{thebibliography}{99}

\bibitem{Newman} Donald J. Newman. A Problem Seminar. Springer - Verlag.1982.

\bibitem{angem1kurs} Аналитическая геометрия на плоскости / Е.М. Семенов, С.Н. Уксусов. – Воронеж : Воронежский государственный университет, 2016. – 100с.

\bibitem{Buhshtab} Теория чисел / А.А. Бухштаб. --- М., 1966.

\bibitem{Ramanujan} Highly composite numbers. Proceedings of the London Mathematical Society, 2, XIV, 1915, 347 – 409 (http://ramanujan.sirinudi.org/Volumes/published/ram15.pdf)

\end{thebibliography}

\end{document}
